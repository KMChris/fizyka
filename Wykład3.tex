\section{Stany kwantowe}

\subsection{Właściwości światła}
Czarne ciało, gdy jest zimne, pochłania wszystkie barwy światła, ale gdy jest bardzo podgrzane, to świeci na biało.

Światło jest systemem \textbf{anomnym}.

\subsection{Światło jako fala}
\[
A(\vec{r}, t) = A_1(\vec{r}, t) + A_2(\vec{r}, t)
\]

Natężenie światła wyraża się wzorem:
\[
I(\vec{r}, t) = |A(\vec{r}, t)|^2
\]

\subsection{Elektron jako fala}
Elektron również może być zapisany jako fala:
\[
\Psi(x,y,z,t) \sim A(\vec{r}, t)
\]
gdzie $A(\vec{r},t)$ jest amplitudą fali elektronowej.

\subsection{Interpretacja funkcji falowej}
\[
|\Psi(x,y,z,t)|^2
\]
jest \textbf{prawdopodobieństwem} znalezienia cząstki w danym obszarze przestrzeni w danym czasie.

Dodatkowo:
- $D$ - bok
- $\lambda$ - puls częsta
- $\lambda$ - złota zanieczyszczona

\subsection{Zasada superpozycji}
Obowiązuje zasada superpozycji (tak jak dla światła):

\[
\Psi = \Psi_A + \Psi_B
\]

ale:
\[
|\Psi_A + \Psi_B|^2 \neq |\Psi_A|^2 + |\Psi_B|^2
\]

\subsection{Interpretacja fali elektronowej}
Mamy jeden elektron, tzn. mamy jeden sygnał, więc w danym obszarze go zaobserwujemy.

Przyjmujemy, że w całej przestrzeni znalezienie elektronu jest równe 1, ponieważ mamy dokładnie jeden elektron:

\[
\int |\Psi(\vec{r}, t)|^2 dV = 1
\]

Szukamy znormowanej zadanej funkcji $\Psi$, tzn. dzielimy każdą funkcję przez jej normę, aby uzyskać 1.

\subsection{Zasada superpozycji w mechanice kwantowej}
\[
\Psi = c_1 \Psi_1 + c_2 \Psi_2
\]

\[
\Psi_1 = |\Psi_1| e^{i \delta_1}
\]

\[
\Psi_2 = |\Psi_2| e^{i \delta_2}
\]

\[
|\Psi|^2 = c_1^2 |\Psi_1|^2 + c_2^2 |\Psi_2|^2 + 2 \Re (c_1 c_2^* |\Psi_1| |\Psi_2| e^{i (\delta_1 - \delta_2)})
\]

To definiuje interferencję.

\subsection{Fala de Broglie'a}
\[
E = h\nu, \quad E = \hbar \omega
\]
\[
p = \frac{h}{\lambda}, \quad p = \hbar k
\]

Gdzie:
- $E$ to energia,
- $\nu$ to częstotliwość,
- $\lambda$ to długość fali,
- $p$ to pęd cząstki.

\subsection{Fala płaska}
Równanie fali płaskiej:
\[
\Psi(x,y,z,t) = A \exp \left( i \left( kx - \omega t \right) \right)
\]
Można również zapisać jako:
\[
\Psi = A \exp \left( \frac{i}{\hbar} (p x - E t) \right)
\]

\subsection{Najprostszy obiekt fali}
- Stojąca fala może się zdarzyć, że nie będzie płaska.
- Stojąca jednowymiarowa fala jest płaska.
- Warunek brzegowy przestrzeni może być dobrany na nasz typ układu.

\subsection{Opis 3D}
Dla trzech wymiarów zapisujemy:
\[
\Psi(\vec{r}, t) = A \exp \left( \frac{i}{\hbar} (\vec{p} \cdot \vec{r} - E t) \right)
\]

Pęd jest opisany jako:
\[
p^2 = k^2
\]

\subsection{Operator pędu i energii}
Operator pędu w kierunku $x$:
\[
\hat{p}_x = -i\hbar \frac{\partial}{\partial x}
\]

Operator energii:
\[
i\hbar \frac{\partial}{\partial t} \Psi = E \Psi
\]

Operator pędu w trzech wymiarach:
\[
\hat{\vec{p}} = -i\hbar \nabla
\]

\subsection{Pakiety falowe}
Zapiszmy jednowymiarową falę cząstki:

\[
\Psi(x,t) = (2\pi\hbar)^{-1/2} \int \phi(p_x) e^{\frac{i}{\hbar} (p_x x - E t)} dp_x
\]

Jest to \textbf{fala płaska}.

\subsubsection{Transformacja Fouriera}
Funkcję falową w przestrzeni $p_x$ możemy zapisać jako:

\[
\phi(p_x) = (2\pi\hbar)^{-1/2} \int e^{-\frac{i}{\hbar} p_x x} \Psi(x) dx
\]

Jest to transformata Fouriera.

\subsubsection{Interpretacja}
Funkcja $\phi(p_x)$ opisuje rozkład prawdopodobieństwa w przestrzeni pędu.

\[
\int dp_x |\phi(p_x)|^2 = 1
\]

co oznacza, że prawdopodobieństwo znalezienia danej wartości $p_x$ jest znormalizowane.

\subsubsection{Energia fali}
Energia kinetyczna:

\[
E = \frac{p_x^2}{2m}
\]

\subsubsection{Własności pakietów falowych}
- Fala po rozmyciu, w którym kontrolujemy szerokość, to \textbf{superpozycja różnych częstości}.
- Determinuje interferencję oraz precyzję pomiaru.

\subsection{Pakiet Gaussowski}
Funkcja $\phi(p_x)$ dla pakietu Gaussowskiego ma postać:

\[
\phi(p_x) = C \exp \left( -\frac{(p_x - p_0)^2}{2 (\Delta p_x)^2} \right)
\]

gdzie $\Delta p_x$ oznacza niepewność pędu.

\subsubsection{Normalizacja}
\[
\int |\phi(p_x)|^2 dp_x = 1
\]

Współczynnik normalizacyjny:
\[
C = \frac{1}{(2\pi\hbar)^{1/4} (\Delta p_x)^{1/2}}
\]

\subsubsection{Pakiet w przestrzeni rzeczywistej}
\[
\Psi(x) = \left( \frac{1}{\pi (\Delta x)^2} \right)^{1/4} e^{-\frac{x^2}{2 (\Delta x)^2}}
\]

Związek nieoznaczoności Heisenberga:

\[
\Delta x \Delta p_x \geq \frac{\hbar}{2}
\]

\subsubsection{Interpretacja}
- Jeśli pakiet jest dobrze zlokalizowany (krótki) w przestrzeni, to jest źle zlokalizowany w przestrzeni pędu.
- Dla podstawowego stanu nie uwzględniamy elektronów.

\subsection{Ewolucja w czasie}

Energia wyrażona przez pęd:
\[
E = \frac{p_x^2}{2m}
\]

Funkcja falowa w czasie:
\[
\Psi(x,t) = (2\pi\hbar)^{-1/4} e^{i p_0 x / \hbar} e^{-i E t / \hbar}
\]

Niepewności pędu i położenia:
\[
\Delta x \Delta p_x = \frac{\hbar}{2}
\]

Szczególny przypadek:
\[
\Delta X \cdot C = \frac{\hbar}{2} p_x
\]

Niepewność Heisenberga:
\[
\Delta x \Delta p \geq \hbar
\]

To jest grube przybliżenie, ponieważ rzeczywistość wymaga bardziej dokładnych obliczeń.

\subsection{Niepewność i interferencja}

Jeśli suponować interferencję, to będzie wymiarowana z elektronem i obiektem.

Niepewności:
\[
\Delta y \Delta p_y \geq \hbar
\]
\[
\Delta x \Delta p_x \geq \hbar
\]

\subsubsection{Para czas-energia}
Transformata Fouriera:
\[
\Psi(t) = \frac{1}{\sqrt{2\pi}} \int g(\omega) e^{-i\omega t} d\omega
\]

Związek nieoznaczoności:
\[
\Delta t \Delta E \geq \hbar
\]

Zależność energii od częstotliwości:
\[
E = \hbar \omega
\]

Graficzna ilustracja nieoznaczoności czasu i energii pokazuje, że krótkie impulsy prowadzą do szerokiego rozkładu częstotliwości.

\subsection{Równanie Schrödingera}

\subsubsection{Motywacja}
Chcemy znaleźć równanie, które będzie opisywało ewolucję fali.

Własności funkcji falowej:
\begin{itemize}
    \item Jeśli $\Psi_1$ i $\Psi_2$ są rozwiązaniami, to kombinacja liniowa $c_1 \Psi_1 + c_2 \Psi_2$ również jest rozwiązaniem.
    \item $\Psi(x,t)$ jest funkcją zespoloną.
\end{itemize}

\subsubsection{Fala płaska}
\[
\Psi(x,t) = A e^{i(px - Et)/\hbar}
\]

Podstawiając zależności:
\[
\hat{p} = -i\hbar \frac{\partial}{\partial x}, \quad \hat{E} = i\hbar \frac{\partial}{\partial t}
\]

Dla swobodnej cząstki:
\[
i\hbar \frac{\partial}{\partial t} \Psi(x,t) = -\frac{\hbar^2}{2m} \frac{\partial^2}{\partial x^2} \Psi(x,t)
\]

\subsubsection{Równanie Schrödingera z potencjałem}
W obecności potencjału $V(x,t)$ równanie przyjmuje postać:
\[
i\hbar \frac{\partial}{\partial t} \Psi(x,t) = \left[ -\frac{\hbar^2}{2m} \nabla^2 + V(x,t) \right] \Psi(x,t)
\]
