\section{Wykład 3}

\textbf{Ciało czarne}, gdy jest zimne, pochłania wszystkie barwy (światło), ale gdy jest bardzo podgrzane, to świeci na biało.
Słońce jest ciałem czarnym.

\subsection{Światło jako fala}
Przykład dla dwóch szczelin:
\[
A(\vec{r}, t) = A_1(\vec{r}, t) + A_2(\vec{r}, t)
\]

Natężenie światła wyraża się wzorem:
\[
I(\vec{r}, t) = |A(\vec{r}, t)|^2
\]

\subsection{Elektron jako fala}
Elektron również może być zapisany jako fala:
\[
\Psi(x,y,z,t) \sim A(\vec{r}, t)
\]
gdzie $A(\vec{r},t)$ jest amplitudą fali elektronowej.

$|\Psi(\Sigma, t)|^2$ jest \textbf{prawdopodobieństwem} znalezienia cząstki w danym obszarze przestrzeni w danym czasie.
Przy czym: 0 - brak cząstki, 1 - jakaś cząstka została zarejestrowana.

Obowiązuje zasada superpozycji (tak jak dla światła):
\[
\Psi = \Psi_A + \Psi_B \quad P \sim |\Psi_A + \Psi_B|^2 \neq |\Psi_A|^2 + |\Psi_B|^2
\]

\subsection{Interpretacja fali elektronowej}
Mamy jeden elektron, tzn. mamy jeden sygnał, że w danym czasie go zaobserwujemy.
Prawdopodobieństwo, że w całej przestrzeni znajdziemy elektron jest równe 1.
\[
\int |\Psi(\vec{r}, t)|^2 dr = 1 = \int \Psi \Psi^* d\vec{r}
\]
Wyrażenie może być ciągłe (np. dla fali) lub dyskretne.
Szukamy znormalizowanej funkcji $\Psi$, tzn. $\Psi$ dzielimy/mnożymy, aby uzyskać 1.

Zasada superpozycji:
\begin{align*}
\Psi &= c_1 \Psi_1 + c_2 \Psi_2\\
\Psi_1 &= |\Psi_1| e^{i \delta_1}\\
\Psi_2 &= |\Psi_2| e^{i \delta_2}\\
|\Psi|^2 &= c_1^2 |\Psi_1|^2 + c_2^2 |\Psi_2|^2 + 2 \Re (c_1 c_2^* |\Psi_1| |\Psi_2| e^{i (\delta_1 - \delta_2)})
\end{align*}
To definiuje interferencję.

\subsection{Fala de Broglie'a}
\[
E = h\nu, \quad E = \hbar \omega
\]
\[
p = \frac{h}{\nu}, \quad p = \hbar k, \quad k = \frac{2\pi}{\lambda}
\]
Gdzie:
\begin{itemize}
    \item $E$ to energia,
    \item $\nu$ to częstotliwość,
    \item $\lambda$ to długość fali,
    \item $p$ to pęd cząstki,
    \item $k$ jest wektorem, który opisuje kierunek i długość fali.
\end{itemize}

\subsection{Fala płaska}
Równanie fali płaskiej:
\[
\Psi(x,y,z,t) = A \exp \left( i \left( kx - \omega t \right) \right)
\]
Można również zapisać jako:
\[
\Psi = A \exp \left( \frac{i}{\hbar} (p_x x - E(p_x) t) \right)
\]
\begin{itemize}
    \item W zależności od położenia rzeczywista część to cosinus i to jest zwykła fala.
    \item Najprostszy obiekt jaki możemy mieć.
    \item Stojąca fala może się zdarzyć, że nie będzie płaska.
    \item Stojąca jednowymiarowa fala jest płaska.
    \item Kierunek przestrzeni może być dowolny, nie musi to być $x$.
\end{itemize}

Dla trzech wymiarów zapisujemy:
\[
\Psi(\vec{r}, t) = A \exp \left( \frac{i}{\hbar} (\vec{p} \cdot \vec{r} - E(p) t) \right)
\]

Pęd jest opisany jako:
\[
\vec{p} \equiv \hbar \vec{k}
\]

Dużym problemem jest całka po całej przestrzeni, bo jest nieskończona.

$\partial_x$: $-i\hbar \frac{\partial}{\partial x} \Psi = p_x \Psi$,

$\partial_t$: $i\hbar \frac{\partial}{\partial t} \Psi = E \Psi$

Operator pędu:
\[
\vec{P}_0 = -i\hbar \vec{\nabla}
\]

\subsection{Pakiety falowe}
Zamiast jednej fali zbiór fal (jesteśmy w jednym wymiarze):
\[
\Psi(x,t) = (2\pi\hbar)^{-1/2} \int_{-\infty}^{+\infty} \phi(p_x) e^{\frac{i}{\hbar} (p_x x - E(p_x) t)} dp_x
\]
Wyrażenie pod całką to \textbf{fala płaska}, a $\phi(p_x)$ to funkcja określająca pakiet falowy.

Rozważmy $t = 0$, wtedy funkcję falową w przestrzeni położenia ma postać:
\[
\Psi(x,0) = (2\pi\hbar)^{-1/2} \int_{-\infty}^{+\infty} \phi(p_x) e^{\frac{i}{\hbar} p_x x} dp_x
\]

Funkcję falową w przestrzeni pędu możemy zapisać jako:
\[
\phi(p_x) = (2\pi\hbar)^{-1/2} \int e^{-\frac{i}{\hbar} p_x x} \Psi(x) dx
\]
Jest to transformata Fouriera.

Niech $\Psi'(x) = (2\pi\hbar)^{-1/2} e^{\frac{i}{\hbar} p_x x}$, wtedy:
\begin{align*}
\phi(p_x) &= (2\pi\hbar)^{-1/2} \int e^\frac{-ip_x x}{\hbar} \cdot e^\frac{ip'_x x}{\hbar} dx \\
&= (2\pi\hbar)^{-1/2} \int e^\frac{i(p'_x - p_x) x}{\hbar} dx \\
&= \delta(p'_x - p_x)
\end{align*}

Fala na przykład po wrzuceniu kamienia do wody to superpozycja różnych częstości.

\[
\int |\delta(p'_x - p_x)|^2 dp_x = \delta(0)
\]

\subsection{Pakiet Gaussowski}
Funkcja $\phi(p_x)$ dla pakietu Gaussowskiego ma postać:
\[
\phi(p_x) = C \exp \left( -\frac{(p_x - p_0)^2}{2 (\Delta p_x)^2} \right)
\]

gdzie $\Delta p_x$ oznacza szerokość pakietu, a $p_0$ to środek pakietu.

\[
\int |\phi(p_x)|^2 dp_x = 1 = |C|^2 \pi^{1/2} (\Delta p_x)
\]
Stąd:
\[
C = \pi^{-\frac{1}{4}} \frac{1}{\sqrt{\Delta p_x}}
\]

\[
\int e^{-\alpha/\mu^2} e^{-\beta \mu^2} d\mu = \left(\frac{\pi}{\alpha}\right)^\frac{1/2} \exp{\frac{\beta^2}{4\alpha}}
\]

\begin{align*}
\Psi(x) &= (2\pi\hbar)^{-1/2} \int e^{\frac{i}{\hbar} (p_x x - E(p_x) t)} \phi(p_x) dp_x \\
&= \cdots \\
&= \pi^{-1/4} \hbar^{-1/2} (\Delta p_x)^{-1/2} e^{\frac{ip_0 x}{\hbar} e^{-(\Delta p_x)^2 x^2}{2\hbar^2}}
\end{align*}

\[
(\frac{(\Delta p_x)^2}{\hbar^2}) = \frac{1}{(\Delta x)^2} \quad \Delta x \Delta p_x = \hbar
\]
\begin{itemize}
    \item Jeśli pakiet jest dobrze zlokalizowany (wąski) w przestrzeni, to jest źle zlokalizowany w przestrzeni pędu.
    \item Jeśli jest nieskończenie szeroki to nie znajdziemy elektronu.
\end{itemize}

\subsection{Ewolucja w czasie}

Energia wyrażona przez pęd:
\[
E = \frac{p_x^2}{2m}
\]

Funkcja falowa w czasie:
\begin{align*}
\Psi(x,t) &= \left(2\pi\hbar \right)^{-1/2} \int e^{i\frac{p_x x - E(p_x)t}{\hbar}} \phi(p_x) dp_x \\
&= \pi^{-\frac{1}{4}}\left[\frac{\frac{\Delta p x}{\hbar}}{1+i\frac{(\Delta p_x)^2 t}{m\hbar}}\right] \exp\left[\frac{\frac{ip_0 x}{\hbar}-\left(\frac{\Delta p_x}{\hbar}\right)^2\frac{x^2}{2}-\frac{ip_0 t}{2x\hbar}}{1+i(\Delta p_x)^2 \frac{t}{m\hbar}}\right]
\end{align*}

\[
|\Psi(x,t)|^2 = \pi^{-\frac{1}{2}}\left[\frac{\frac{\Delta p x}{\hbar}}{\left[1+i\frac{(\Delta p_x)^4 t^2}{m^2\hbar^2}\right]^{\frac{1}{2}}}\right] \exp\left[\frac{-\left(\frac{\Delta p_x}{\hbar}\right)^2(x-Vgt)^2}{1+\frac{(\Delta p_x)^4 t^4}{m^2\hbar^2}}\right]
\]

Prędkość grupowa:
\[
Vg = \frac{p_0}{m}
\]
Rozważamy szczególny przypadek.
\[
\Delta x (t) = \frac{\hbar}{\Delta p_x} \underbrace{\left[1 + \frac{(\Delta p_x)^4}{m^2\hbar^2}t^2\right]^{1/2}}_{B}
\]
Zawsze $B \geq 1$.
\[
\Delta x \Delta p = \hbar B
\]

Nierówność Heisenberga:
\[
\Delta x \Delta p \geq \hbar
\]

To jest grube przybliżenie, ponieważ rzeczywistość wymaga bardziej dokładnych obliczeń.

Interpretacja $x$ to błąd wymiaru.
\[
\Delta y \Delta p_y \geq \hbar
\]
\[
\Delta z \Delta p_z \geq \hbar
\]

\subsection{Para czas/energia}
Transformata Fouriera:
\[
\begin{cases}
\Psi(t) = \frac{1}{\sqrt{2\pi}} \int G(\omega) e^{-i\omega t} d\omega\\
G(\omega) = \frac{1}{\sqrt{2\pi}} \int \Psi(t) e^{i\omega t} dt
\end{cases}
\]

Stąd
\[
\Delta \omega \Delta t \geq 1
\]

Związek nieoznaczoności:
\[
\Delta E \Delta t \geq \hbar
\]

Zależność energii od częstotliwości:
\[
E = \hbar \omega
\]

\subsection{Równanie Schrödingera}

\textbf{Motywacja}: chcemy znaleźć równanie, które będzie opisywało ewolucję fali.

\[
\begin{cases}
    \Psi_1 \text{ -- rozwiązanie} \\
    \Psi_2 \text{ -- rozwiązanie}
\end{cases} \Rightarrow c_1 \Psi_1 + c_2 \Psi_2 \text{ -- rozwiązanie}
\]
Rozwiązanie równania Schrödingera jest liniowe. $\Psi$ musi posiadać pierwszą pochodną.

Fala płaska:
\[
\Psi(x,t) = A e^{\frac{i(px x - Et)}{\hbar}}
\]

$\frac{\partial}{\partial x}$ $-i \hbar \frac{\partial}{\partial x} \Psi = p_x \Psi \xRightarrow{\frac{\partial}{\partial x}} \frac{\partial^2\Psi}{\partial x^2} = -\frac{p^2}{\hbar^2} \Psi = -\frac{2mE}{\hbar^2} \Psi$

$\frac{\partial}{\partial t}$ $\frac{\partial\Psi}{\partial t} = \frac{-iE\Psi}{\hbar}$

\[
-i\hbar \frac{\partial}{\partial t} \Psi(x, t) = -\frac{-\hbar^2}{2m} \frac{\partial^2}{\partial x^2} \Psi(x, t)
\]

Interpretacja:
\[
-i\hbar \frac{\partial}{\partial x} \sim p_x \Rightarrow \frac{-\hbar^2}{2m} \frac{\partial^2}{\partial x^2} \sim E_{\text{kin}}
\]

Gdy dodamy potencjał $V(x,t)$:
\[
-i\hbar \frac{\partial}{\partial t} \Psi(x,t) = \left[ -\frac{\hbar^2}{2m} \frac{\partial^2}{\partial x^2} + V(x,t) \right] \Psi(x,t)
\]
Gdy $V(x,t) = 0$ to mamy rozwiązanie.
W trzech wymiarach:
\[
i\hbar \frac{\partial}{\partial t} \Psi(\vec{r},t) = \left[ -\frac{\hbar^2}{2m} \nabla^2 + V(\vec{r},t) \right] \Psi(\vec{r},t)
\]
Jest to równanie Schrödingera.
