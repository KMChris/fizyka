\documentclass[a4paper,12pt]{article}
\usepackage[T1]{fontenc}
\usepackage[polish]{babel}
\usepackage{amsmath, amssymb}
\usepackage{mathtools}
\usepackage{physics}
\usepackage{graphicx}
\usepackage{float}
\usepackage{geometry}
\usepackage{parskip}
\usepackage{tikz}
\usepackage{amsmath}
\usepackage{braket}
\usepackage[hidelinks]{hyperref}
\geometry{left=25mm, right=25mm, top=25mm, bottom=25mm}
\graphicspath{{images/}{./}}

\begin{document}

% temporary
\hbadness=10000 % suppress underfull hbox warnings
\vbadness=10000 % suppress underfull vbox warnings
\hfuzz=1000pt % suppress overfull hbox warnings (visual + log)
\vfuzz=1000pt % suppress overfull vbox warnings (visual + log)
\overfullrule=0pt % suppress black bars in output

\title{Podstawy mechaniki kwantowej}
\author{Notatki z wykładu}
\date{\today}
\maketitle
\tableofcontents

\newcommand{\RR}{\mathbb R}
\newcommand{\CC}{\mathbb C}

\section{Historia powstania fizyki kwantowej}

\subsection{Zapomnijmy o mechanice klasycznej}
Związek z nią będzie jasny, kiedy pójdziemy głębiej w teorię.

\subsection{Promieniowanie ciała doskonale czarnego}
Eksperyment Stefana-Boltzmanna (1878) badał promieniowanie cieplne emitowane przez ciało doskonale czarne.
Ciało doskonale czarne to obiekt, który pochłania całe promieniowanie i emituje je zgodnie z temperaturą.
\begin{figure}[H]
    \centering
    \includegraphics[width=0.3\textwidth]{blackbody}
    \caption{Ciało doskonale czarne. \textit{Źródło: Wikipedia}}
    \label{fig:blackbody}
\end{figure}

Pokazano, że całkowita energia wypromieniowywana przez takie ciało jest proporcjonalna do czwartej potęgi jego temperatury absolutnej
\begin{equation*}
    R(T) = \sigma T^4,
\end{equation*}
gdzie $R$ to moc promieniowania na jednostkę powierzchni, $T$ to temperatura w kelwinach, a $\sigma$ to stała Stefana-Boltzmanna.

Całkowita moc promieniowania to
\begin{equation*}
    R(T) = \int_0^\infty \rho(\lambda, T) d\lambda,
\end{equation*}
gdzie $\lambda$ to długość fali, a $\rho(\lambda, T)$ to spektralna funkcja rozkładu.

W 1893 Wien zauważył, że spektralna gęstość promieniowania nie zależy od $\lambda$ i $T$ osobno, ale od ich iloczynu $\lambda T$
\begin{equation*}
    \rho(\lambda, T) = \lambda^{-5} f(\lambda T).
\end{equation*}

\subsection{Prawo Rayleigha-Jeansa}
W klasycznej elektrodynamice, promieniowanie elektromagnetyczne opisane jako fale stojące daje rozkład energii w funkcji długości fali.
Liczba takich fal o długości od $\lambda$ do $\lambda + d\lambda$ to
\begin{equation*}
    \rho(\lambda, T) = \frac{8\pi}{\lambda^4} \cdot \bar{\epsilon},
\end{equation*}
gdzie $\bar{\epsilon}$ to średnia energia takiej fali.
Wzór ten jest dokładny dla długich fal, ale prowadzi do problemu z ,,katastrofą ultrafioletową'' przy krótkich falach, co zostało skorygowane przez teorię kwantową Plancka.

\begin{figure}[H]
    \centering
    \includegraphics[width=0.5\textwidth]{widmo-promieniowania}
    \caption{Widmo promieniowania ciała doskonale czarnego w wybranych temperaturach. \textit{Źródło: Zbigniew Kąkol, Jan Żukrowski (e-Fizyka, AGH)}}
    \label{fig:widmo-promieniowania}
\end{figure}

\subsection{Teoria kwantowa Plancka}
W 1900 roku Planck zaproponował, że ciała emitują światło w postaci kwantów ($\epsilon = n\epsilon_0$)
\begin{equation*}
    \bar{\epsilon} = \frac{\sum\limits_{n=0}^{\infty} n\epsilon_0 \exp(-\frac{n\epsilon_0}{kT})}{\sum\limits_{n=0}^{\infty} \exp(-\frac{n\epsilon_0}{kT})} = \cdots = \frac{\epsilon_0}{\exp(\frac{\epsilon_0}{kT}) - 1},
\end{equation*}
gdzie \(\epsilon_0 = h \nu = \frac{h c}{\lambda}\) jest energią jednego kwantu promieniowania.

Z tego wyrażenia Planck otrzymał rozkład promieniowania w funkcji długości fali, który ma postać
\begin{equation*}
    \beta(\lambda, T) = \frac{8\pi hc}{\lambda^5} \cdot \frac{1}{\exp(\frac{hc}{k\lambda T}) - 1},
\end{equation*}
Wzór ten zgadza się z wynikami eksperymentalnymi, eliminując problem ,,katastrofy ultrafioletowej''.

\subsection{Efekt fotoelektryczny}
Efekt fotoelektryczny to zjawisko emisji elektronów z powierzchni metalu pod wpływem padającego na niego światła.

\begin{figure}[H]
    \centering
    \includegraphics[width=0.5\textwidth]{efekt-fotoelektryczny}
    \caption{Układ do obserwacji zjawiska fotoelektrycznego. \textit{Źródło: Zbigniew Kąkol, Jan Żukrowski (e-Fizyka, AGH)}}
    \label{fig:efekt-fotoelektryczny}
\end{figure}

W 1900 roku doświadczenia Lenarda wykazały, że energia elektronów zależy od częstotliwości światła, a nie jego intensywności.
Einstein sformułował wzór efektu fotoelektrycznego
\begin{equation*}
    \frac{1}{2} m v_{\max}^2 = h\nu - W,
\end{equation*}
gdzie $W$ to funkcja pracy metalu (zależna od rodzaju metalu).

\subsection{Widma atomowe i model Bohra}
Newton (1660) badał rozszczepienie światła. Melvill (1755) odkrył, że różne pierwiastki mają charakterystyczne linie widmowe. Kirchhoff (1855) zauważył, że widmo zależy od typu atomu i istnieją zarówno widma emisyjne, jak i absorpcyjne.

Balmer (1885) podał wzór:
\begin{equation*}
    \lambda = C \cdot \frac{n^2}{n^2 - 4}.
\end{equation*}

Rydberg sformułował bardziej ogólny wzór:
\begin{equation*}
    \tilde{\nu} = R_H \left( \frac{1}{2^2} - \frac{1}{n^2} \right).
\end{equation*}

\section{Funkcja falowa}

\subsection{Eksperyment z dwoma szczelinami}

Wobrażmy sobie ścianę z dwoma wąskimi otworami oraz drugą równoległą ścianę za nią, która nie ma żadnych otworów.
Teraz wyobraźmy sobie, że osoba strzela kulami we wszystkich kierunkach. Większość kul zatrzymuje się na pierwszej ścianie,
lecz część kul przechodzi przez otwory i trafia na drugą ścianę.
Jakiego obrazu spodziewamy się na drugiej ścianie? Spodziewamy się dwóch kropek, w miejscach odpowiadających otworom na pierwszej ścianie. To też obserwujemy.

\begin{figure}[H]
    \centering
    \includegraphics[width=0.5\textwidth]{szczeliny}
    \caption{Eksperyment z dwoma szczelinami. \textit{Źródło: Ranjbar, Vahid. (2023)}}
    \label{fig:szczeliny}
\end{figure}

\subsection{Eksperyment ze światłem}
W roku 1801 Thomas Young przeprowadził podobny eksperyment, ale przepuszczając przez szczeliny światło.
Przez dwie szczeliny przechodziła tak zwana fala płaska, poruszająca się w kierunku ekranu.
W sensie optyki klasycznej albo termodynamiki klasycznej, możemy powiedzieć, że przykładowo światło słoneczne jest taką falą płaską.
Ta fala płaska przechodzi przez szczeliny, a następnie dalej jako fala płaska przemieszcza się w kierunku oddalonego ekranu.

Co zobaczymy na ekranie? Na ekranie zobaczymy coś niespodziewanego - będzie to obraz interferencyjny.
\begin{figure}[H]
    \centering
    \includegraphics[width=0.5\textwidth]{szczeliny-fale}
    \caption{Eksperyment z dwoma szczelinami. \textit{Źródło: e-Fizyka, AGH}}
    \label{fig:szczeliny-fale}
\end{figure}

Zastanówmy się, w jaki sposób można opisać intensywności światła ukazujące się na ekranie.

Zacznijmy od amplitudy fali (amplitudy światła) - jest to wektor zależny od położenia w~przestrzeni oraz czasu:
\begin{equation*}
    A(\bar{r}, t)
\end{equation*}
Intensnywność światła $I$ można zapisać jako kwadrat amplitudy niezależnej od modułu:
\begin{equation*}
    I = |A|^2
\end{equation*}
Następnie pojawia się tak zwana zasada superpozycji. Aby obliczyć amplitudę całkowitą, musimy zsumować amplitudy fal pochodzących z obu szczelin (z obu źródeł):
\begin{equation*}
    \bar{A}(\bar{r}, t) = \bar{A}_1(\bar{r}, t) + \bar{A}_2(\bar{r}, t)
\end{equation*}

Intensywność całkowita będzie przybierać następującą postać:
\begin{equation*}
    I = |A_1|^2 + |A_2|^2 + A_1 A_2^* + A_1^* A_2
\end{equation*}

Człon $A_1 A_2^* + A_1^* A_2$ jest odpowiedzialny za interferencję. Obraz widoczny na ekranie jest spowodowany superpozycją fal pochodzących z obu szczelin.

\subsection{Proste zagadnienie}

Rozważmy najprostsze zagadnienie. W tym zagadnieniu podkreślamy, że odległość między szczelinami $d$ jest mała (dużo mniejsza niż odległość do ekranu $D$).
Na ekranie zaznaczamy pewien punkt $x$ oraz zaznaczamy odległości punktu $x$ od szczelin. Odległość $x$ od szczeliny $1$ wynosi $r_1$, a odległość $x$
od szczeliny $2$ wynosi $r_2$.

\begin{figure}[H]
    \centering
    \includegraphics[width=0.5\textwidth]{prosty-eksperyment}
    \caption{Prosty eksperyment.}
    \label{fig:prosty-eksperyment}
\end{figure}

Rozważamy proste fale monochromatyczne, to znaczy amplitudy dla nich mają następującą postać:
\begin{equation*}
    A_1 = a_1 \exp[i(\omega t - \bar{k} \bar{r}_1 + \delta_1)]
\end{equation*}
\begin{equation*}
    A_2 = a_2 \exp[i(\omega t - \bar{k} \bar{r}_2 + \delta_2)]
\end{equation*}

Ponieważ mamy jedno źródło światła, możemy przyjąć, że $a_1 = a_2 = a$ i $\delta_1 = \delta_2 = 0$. 
Symbol $k$ oznacza wektor falowy, który ma kierunek fali. Zapisujemy go jako:
\begin{equation*}
    k = \frac{2\pi}{\lambda}
\end{equation*}
Wersor kierunku fali zapisujemy jako:
\begin{equation*}
    \frac{\bar{k}}{|k|}
\end{equation*}
Przechodzimy do geometrii. Chcemy zrozumieć, jaka będzie intensywność w punkcie $x$ na ekranie.

Wektory $\bar{k}_1$ i $\bar{k}_2$ będą równoległe do siebie, zatem możemy zapisać: $\bar{k}_1 = \bar{k}_2$.


Wyznaczamy $r_1^2$ i $r_2^2$:
\begin{equation*}
    r_1^2 = D^2 + \left(x + \frac{d}{2} \right)^2
\end{equation*}
\begin{equation*}
    r_2^2 = D^2 + \left(x - \frac{d}{2} \right)^2
\end{equation*}
Stąd:
\begin{equation*}
    r_1^2 - r_2^2 = 2xd
\end{equation*}
Ponieważ $r_1$ i $r_2$ są bardzo duże, a różnica między nimi jest mała, możemy zapisać:
\begin{equation*}
    r_1 - r_2 \approx \frac{xd}{D}
\end{equation*}
Intensywność końcowa:
\begin{align*}
    I &= \left(a\cdot e^{i\omega t}\right)^2 \cdot \left[e^{-ikr_1}+e^{-ikr_2}\right]^2 \\
    &= 2 a^2 \left(\cos{\left(kr_1-kr_2\right)}+1\right) \\
    &= 2 a^2 \left(1 + \cos{\left(k(r_1-r_2)\right)}\right) \\
    &= 2 a^2 \left(1 + \cos{\left(\frac{2\pi}{\lambda}\cdot\frac{xd}{D}\right)}\right) = I(x)
\end{align*}

Pytanie - dla jakich $x$ będzie maksymalna intensywność?
Aby znaleźć maksimum, obliczamy pochodną $I(x)$ i przyrównujemy do zera - wartość w zerze będzie albo maksimum, albo minimum.
Chcemy zatem, aby argument cosinusa przyjmował wartość $1$. Będzie to dla $2k\pi$, $k=0,1,2,3,\ldots$. Przyrównując:
\begin{equation*}
    \frac{2\pi}{\lambda} \frac{xd}{D} = 2k\pi
\end{equation*}
Rozwiązując dla $x$ otrzymujemy:
\begin{equation*}
    x_{max} = k \frac{\lambda D}{d}
\end{equation*}
W ten sposób możemy wytłumaczyć obraz interferencyjny - pojawiają się punkty o maksymalnej intensywności
rozłożone wzdłuż osi $x$, a odległość między kolejnymi punktami wynosi $\frac{\lambda D}{d}$.

\subsection{Ciało czarne}

\textbf{Ciało czarne}, gdy jest zimne, pochłania wszystkie barwy (światło), ale gdy jest bardzo podgrzane, to świeci na biało.
Słońce jest ciałem czarnym.

\subsection{Światło jako fala}
Przykład dla dwóch szczelin:
\[
A(\vec{r}, t) = A_1(\vec{r}, t) + A_2(\vec{r}, t)
\]
gdzie $A_1(\vec{r}, t)$ i $A_2(\vec{r}, t)$ są amplitudami fal przechodzących przez każdą ze szczelin.

Natężenie światła wyraża się wzorem:
\[
I(\vec{r}, t) = |A(\vec{r}, t)|^2
\]
co oznacza, że natężenie światła w danym punkcie jest proporcjonalne do kwadratu amplitudy fali w tym punkcie.

\subsection{Elektron jako fala}
W mechanice kwantowej elektron $e^-$ jest traktowany jako fala, co jest formalnie określone poprzez funkcję falową. 

Funkcja falowa $\Psi(x,y,z,t)$ opisuje stan elektronu w trójwymiarowej przestrzeni oraz w~czasie. Dla swobodnego elektronu można przyjąć:
\[
\Psi(x,y,z,t) \sim A(\vec{r}, t)
\]
gdzie $A(\vec{r},t)$ jest amplitudą fali elektronowej.

Kwadrat modułu funkcji falowej $|\Psi(\Sigma, t)|^2$ jest \textbf{prawdopodobieństwem} (gęstością prawdopodobieństwa) znalezienia cząstki w danym obszarze przestrzeni w danym czasie.
Przy czym: 0 - brak cząstki, 1 - jakaś cząstka została zarejestrowana. Można go traktować jako ,,intensywność'' fali elektronowej.

Obowiązuje zasada superpozycji (tak jak dla światła).
Zatem jeżeli mamy dwie funkcje falowe, $\Psi_A$ i $\Psi_B$, możemy je zsumować, otrzymując nową funkcję falową, będącą superpozycją tych funkcji.
\[
\Psi = \Psi_A + \Psi_B \quad P \sim |\Psi_A + \Psi_B|^2 \neq |\Psi_A|^2 + |\Psi_B|^2
\]
Prawdopodobieństwo znalezienia elektronu w danym miejscu jest proporcjonalne do kwadratu modułu sumy funkcji falowych,
co oznacza, że prawdopodobieństwo znalezienia elektronu w danym miejscu jest zależne od interferencji fal $\Psi_A$ i $\Psi_B$.

\subsection{Interpretacja fali elektronowej}
Mamy jeden elektron, tzn. mamy jeden sygnał, że w danym czasie go zaobserwujemy.
Całkowite prawdopodobieństwo, że w całej przestrzeni znajdziemy elektron musi być równe 1.
Dlatego funkcja falowa musi być znormalizowana, zatem:
\[
\int |\Psi(\vec{r}, t)|^2 d\vec{r} = 1 = \int \Psi \Psi^* d\vec{r}
\]
Wyrażenie może być ciągłe (np. dla fali) lub dyskretne.
Szukamy znormalizowanej funkcji $\Psi$, tzn. $\Psi$ dzielimy/mnożymy, aby uzyskać 1.

Wróćmy do superpozycji dwóch funkcji falowych. Ogólna funkcja falowa może być liniową kombinacją dwóch funkcji falowych, gdzie $c_1$ i $c_2$ są współczynnikami.
Każda funkcja falowa może być wyrażona w postaci zespolonej, z amplitudą $|\Psi_i|$ oraz fazą $\delta_i$.
Stąd moduł kwadratu funkcji falowej $|\Psi|^2$ daje się rozwinąć do postaci:
\begin{align*}
\Psi &= c_1 \Psi_1 + c_2 \Psi_2\\
\Psi_1 &= |\Psi_1| e^{i \delta_1}\\
\Psi_2 &= |\Psi_2| e^{i \delta_2}\\
|\Psi|^2 &= c_1^2 |\Psi_1|^2 + c_2^2 |\Psi_2|^2 + 2 \Re (c_1 c_2^* |\Psi_1| |\Psi_2| e^{i (\delta_1 - \delta_2)})
\end{align*}
Ten wzór uwzględnia zarówno intensywności poszczególnych fal, jak i interferencję między nimi. Człon odpowiedzialny za interferencję ma postać:
\begin{equation*}
    2 \Re \left[ c_1 c_2^* |\Psi_1| |\Psi_2| e^{i (\delta_1 - \delta_2)} \right]
\end{equation*}

\subsection{Fala de Broglie'a}
\[
E = h\nu, \quad E = \hbar \omega
\]
\[
p = \frac{h}{\lambda}, \quad p = \hbar k, \quad k = \frac{2\pi}{\lambda}
\]
Gdzie:
\begin{itemize}
    \item $E$ to energia,
    \item $\nu$ to częstotliwość,
    \item $\lambda$ to długość fali,
    \item $p$ to pęd cząstki,
    \item $k$ jest wektorem, który opisuje kierunek i długość fali.
\end{itemize}

\subsection{Fala płaska}
Równanie fali płaskiej:
\[
\Psi(x,y,z,t) = A \exp \left( i \left( kx - \omega t \right) \right)
\]
Można również zapisać jako:
\[
\Psi = A \exp \left( \frac{i}{\hbar} (p_x x - E(p_x) t) \right)
\]
\begin{itemize}
    \item W zależności od położenia rzeczywista część to cosinus i to jest zwykła fala.
    \item Najprostszy obiekt jaki możemy mieć.
    \item Stojąca fala może się zdarzyć, że nie będzie płaska.
    \item Stojąca jednowymiarowa fala jest płaska.
    \item Kierunek przestrzeni może być dowolny, nie musi to być $x$.
\end{itemize}

Dla trzech wymiarów zapisujemy:
\[
\Psi(\vec{r}, t) = A \exp \left( \frac{i}{\hbar} (\vec{p} \cdot \vec{r} - E(p) t) \right)
\]

Pęd jest opisany jako:
\[
\vec{p} \equiv \hbar \vec{k}
\]

Dużym problemem jest całka po całej przestrzeni, bo jest nieskończona.

$\partial_x$: $-i\hbar \frac{\partial}{\partial x} \Psi = p_x \Psi$,

$\partial_t$: $i\hbar \frac{\partial}{\partial t} \Psi = E \Psi$

Operator pędu:
\[
\vec{P}_0 = -i\hbar \vec{\nabla}
\]

\subsection{Pakiety falowe}
Zamiast jednej fali zbiór fal (jesteśmy w jednym wymiarze):
\[
\Psi(x,t) = (2\pi\hbar)^{-1/2} \int_{-\infty}^{+\infty} \phi(p_x) e^{\frac{i}{\hbar} (p_x x - E(p_x) t)} dp_x
\]
Wyrażenie pod całką to \textbf{fala płaska}, a $\phi(p_x)$ to funkcja określająca pakiet falowy.

Rozważmy $t = 0$, wtedy funkcję falową w przestrzeni położenia ma postać:
\[
\Psi(x,0) = (2\pi\hbar)^{-1/2} \int_{-\infty}^{+\infty} \phi(p_x) e^{\frac{i}{\hbar} p_x x} dp_x
\]

Funkcję falową w przestrzeni pędu możemy zapisać jako:
\[
\phi(p_x) = (2\pi\hbar)^{-1/2} \int e^{-\frac{i}{\hbar} p_x x} \Psi(x) dx
\]
Jest to transformata Fouriera.

Niech $\Psi'(x) = (2\pi\hbar)^{-1/2} e^{\frac{i}{\hbar} p_x x}$, wtedy:
\begin{align*}
\phi(p_x) &= (2\pi\hbar)^{-1/2} \int e^\frac{-ip_x x}{\hbar} \cdot e^\frac{ip'_x x}{\hbar} dx \\
&= (2\pi\hbar)^{-1/2} \int e^\frac{i(p'_x - p_x) x}{\hbar} dx \\
&= \delta(p'_x - p_x)
\end{align*}

Fala na przykład po wrzuceniu kamienia do wody to superpozycja różnych częstości.

\[
\int |\delta(p'_x - p_x)|^2 dp_x = \delta(0)
\]

\subsection{Pakiet Gaussowski}
Funkcja $\phi(p_x)$ dla pakietu Gaussowskiego ma postać:
\[
\phi(p_x) = C \exp \left( -\frac{(p_x - p_0)^2}{2 (\Delta p_x)^2} \right)
\]

gdzie $\Delta p_x$ oznacza szerokość pakietu, a $p_0$ to środek pakietu.

\[
\int |\phi(p_x)|^2 dp_x = 1 = |C|^2 \pi^{1/2} (\Delta p_x)
\]
Stąd:
\[
C = \pi^{-\frac{1}{4}} \frac{1}{\sqrt{\Delta p_x}}
\]

\[
\int e^{-\alpha/\mu^2} e^{-\beta \mu^2} d\mu = \left(\frac{\pi}{\alpha}\right)^\frac{1/2} \exp{\frac{\beta^2}{4\alpha}}
\]

\begin{align*}
\Psi(x) &= (2\pi\hbar)^{-1/2} \int e^{\frac{i}{\hbar} (p_x x - E(p_x) t)} \phi(p_x) dp_x \\
&= \cdots \\
&= \pi^{-1/4} \hbar^{-1/2} (\Delta p_x)^{-1/2} e^{\frac{ip_0 x}{\hbar} e^{-(\Delta p_x)^2 x^2}{2\hbar^2}}
\end{align*}

\[
(\frac{(\Delta p_x)^2}{\hbar^2}) = \frac{1}{(\Delta x)^2} \quad \Delta x \Delta p_x = \hbar
\]
\begin{itemize}
    \item Jeśli pakiet jest dobrze zlokalizowany (wąski) w przestrzeni, to jest źle zlokalizowany w przestrzeni pędu.
    \item Jeśli jest nieskończenie szeroki to nie znajdziemy elektronu.
\end{itemize}

\subsection{Ewolucja w czasie}

Energia wyrażona przez pęd:
\[
E = \frac{p_x^2}{2m}
\]

Funkcja falowa w czasie:
\begin{align*}
\Psi(x,t) &= \left(2\pi\hbar \right)^{-1/2} \int e^{i\frac{p_x x - E(p_x)t}{\hbar}} \phi(p_x) dp_x \\
&= \pi^{-\frac{1}{4}}\left[\frac{\frac{\Delta p x}{\hbar}}{1+i\frac{(\Delta p_x)^2 t}{m\hbar}}\right] \exp\left[\frac{\frac{ip_0 x}{\hbar}-\left(\frac{\Delta p_x}{\hbar}\right)^2\frac{x^2}{2}-\frac{ip_0 t}{2x\hbar}}{1+i(\Delta p_x)^2 \frac{t}{m\hbar}}\right]
\end{align*}

\[
|\Psi(x,t)|^2 = \pi^{-\frac{1}{2}}\left[\frac{\frac{\Delta p x}{\hbar}}{\left[1+i\frac{(\Delta p_x)^4 t^2}{m^2\hbar^2}\right]^{\frac{1}{2}}}\right] \exp\left[\frac{-\left(\frac{\Delta p_x}{\hbar}\right)^2(x-V_gt)^2}{1+\frac{(\Delta p_x)^4 t^4}{m^2\hbar^2}}\right]
\]

Prędkość grupowa:
\[
V_g = \frac{p_0}{m}
\]
Rozważamy szczególny przypadek.
\[
\Delta x (t) = \frac{\hbar}{\Delta p_x} \underbrace{\left[1 + \frac{(\Delta p_x)^4}{m^2\hbar^2}t^2\right]^{1/2}}_{B}
\]
Zawsze $B \geq 1$.
\[
\Delta x \Delta p = \hbar B
\]

Nierówność Heisenberga:
\[
\Delta x \Delta p \geq \hbar
\]

To jest grube przybliżenie, ponieważ rzeczywistość wymaga bardziej dokładnych obliczeń.

Interpretacja $x$ to błąd wymiaru.
\[
\Delta y \Delta p_y \geq \hbar
\]
\[
\Delta z \Delta p_z \geq \hbar
\]

\subsection{Para czas/energia}
Transformata Fouriera:
\[
\begin{cases}
\Psi(t) = \frac{1}{\sqrt{2\pi}} \int G(\omega) e^{-i\omega t} d\omega\\
G(\omega) = \frac{1}{\sqrt{2\pi}} \int \Psi(t) e^{i\omega t} dt
\end{cases}
\]

Stąd
\[
\Delta \omega \Delta t \geq 1
\]

Związek nieoznaczoności:
\[
\Delta E \Delta t \geq \hbar
\]

Zależność energii od częstotliwości:
\[
E = \hbar \omega
\]

\subsection{Równanie Schrödingera}

\textbf{Motywacja}: chcemy znaleźć równanie, które będzie opisywało ewolucję fali.

\[
\begin{cases}
    \Psi_1 \text{ -- rozwiązanie} \\
    \Psi_2 \text{ -- rozwiązanie}
\end{cases} \Rightarrow c_1 \Psi_1 + c_2 \Psi_2 \text{ -- rozwiązanie}
\]
Rozwiązanie równania Schrödingera jest liniowe. $\Psi$ musi posiadać pierwszą pochodną.

Fala płaska:
\[
\Psi(x,t) = A e^{\frac{i(px x - Et)}{\hbar}}
\]

$\frac{\partial}{\partial x}$ $-i \hbar \frac{\partial}{\partial x} \Psi = p_x \Psi \xRightarrow{\frac{\partial}{\partial x}} \frac{\partial^2\Psi}{\partial x^2} = -\frac{p^2}{\hbar^2} \Psi = -\frac{2mE}{\hbar^2} \Psi$

$\frac{\partial}{\partial t}$ $\frac{\partial\Psi}{\partial t} = \frac{-iE\Psi}{\hbar}$

\[
-i\hbar \frac{\partial}{\partial t} \Psi(x, t) = -\frac{-\hbar^2}{2m} \frac{\partial^2}{\partial x^2} \Psi(x, t)
\]

Interpretacja:
\[
-i\hbar \frac{\partial}{\partial x} \sim p_x \Rightarrow \frac{-\hbar^2}{2m} \frac{\partial^2}{\partial x^2} \sim E_{\text{kin}}
\]

Gdy dodamy potencjał $V(x,t)$:
\[
-i\hbar \frac{\partial}{\partial t} \Psi(x,t) = \left[ -\frac{\hbar^2}{2m} \frac{\partial^2}{\partial x^2} + V(x,t) \right] \Psi(x,t)
\]
Gdy $V(x,t) = 0$ to mamy rozwiązanie.
W trzech wymiarach:
\[
i\hbar \frac{\partial}{\partial t} \Psi(\vec{r},t) = \left[ -\frac{\hbar^2}{2m} \nabla^2 + V(\vec{r},t) \right] \Psi(\vec{r},t)
\]
Jest to równanie Schrödingera.

\section{Stany kwantowe}

\subsection{Eksperyment S-G w interpretacji Feynmana}

Zawartość tego wykładu pochodzi z książki  "'\textit{Feymana wykłady z fizyki} "'. Feyman prowadził te wykłady na początku lat 60 i w głównej mierze zebrały uwagę profesorów gdyż były zbyt ciężkie dla studentów. Przerobimy to w taki sposób, aby to było jasne.
Wprowadźmy oznaczenia. 

\begin{minipage}{0.6 \textwidth}
	\begin{figure}[H]
		\centering
		\includegraphics[width=0.5\textwidth]{eksperymentszczelinami}
		\caption{Eksperyment z dwoma szczelinami. \newline \textit{Źródło: Byju's}}
		\label{fig:szczeliny2}
	\end{figure}
\end{minipage}
\begin{minipage}{0.35 \textwidth}
	Prawdopodobieństwo, że elektron ze stanu S przejdzie do stanu X \newline [Stan $X$] $\leftarrow$ [Stan $S$] = $\vert\langle X\vert S\rangle\vert^2$ będziemy nazywać amplitudą.
\end{minipage}

Amplitudę ze stanu $S$ do stanu $X$ możemy w tym przypadku zapisać dwojako, w postaci $\langle X\vert1\rangle, \langle 1\vert S\rangle$ jako przejścia ze stanu $S$ do stanu $1$ a następnie ze stanu $1$ do stanu $X$, oraz analogicznie w postaci $\langle X\vert2\rangle , \langle 2\vert S\rangle$. Przepiszemy korzystając z tych oznaczeń zasadę superpozycji. Jeżeli chcemy przejść ze stanu $S$ do stanu $X$ to będziemy to zapisywać jako sumę amplitud $\langle X\vert S\rangle = \langle X\vert1\rangle\langle 1\vert S\rangle + \langle X\vert2\rangle \langle 2\vert S\rangle$. Jeżeli chcemy zapisać prawdopodobieństwo to musimy obliczyć z tego moduł do kwadratu i obliczyć to wszystko w liczbach zespolonych $\vert\langle X\vert S\rangle \vert^2 = \vert[...]\vert^2$. 
\textbf{Pytanie} co oznacza superpozycja? Jeżeli chcemy obliczyć amplitudę to jest to suma poszczególnych amplitud. Amplituda $S$ do $Z$ jest superpozycją przejścia przez szczeliny.
\subsection{Urządzenie S-G dla atomów ze spinem 1}
\begin{minipage}{0.2 \textwidth}
	\vspace{11pt}
	Spin oznaczamy
\end{minipage}
\begin{minipage}{0.11 \textwidth}
	\begin{equation*}
		X_{\beta} = 
		\begin{cases}
			+1 \\
			0 \hspace{20pt} \leftarrow \text{ trzy rodzaje kwantowe.} \\
			-1 \\
		\end{cases}
	\end{equation*}
\end{minipage}
\begin{figure}[H]
	\centering
	\includegraphics[width=0.6\textwidth]{urzadzeniesg}
	\caption{Urządzenie Sterna-Gerlacha. \textit{Źródło: PennState ESM}}
	\label{fig:urzSG}
\end{figure}
\begin{figure}[H]
	\centering
	\includegraphics[width=0.8\textwidth]{urzadzeniesgmod.png}
	\caption{Modyfikacja urządzenia Sterna-Gerlacha. \textit{Źródło: ResearchGate}}
	\label{fig:urzSGmod}
\end{figure}
Dlaczego tak robimy? Ponieważ dzięki zmodyfikowaniu urządzenia S-G nasze atomy będą podążały jak na rysunku powyżej do jednego punktu. I to będzie nasze modyfikowane urządzenie S-G o którym dalej będziemy mówić. Następnie możemy rozważyć kolejny schemat. Możemy wstawić płytkę przez którą nie mogą przechodzić atomy, więc na końcu będziemy mieli tylko atomy nie zablokowane.

\begin{minipage}{0.5 \textwidth}
	\begin{figure}[H]
		\centering
		\includegraphics[width=0.8\textwidth]{urzadzeniesgblok}
		\caption{Zablokowane atomy poza  "'$+$ "'}
		\label{fig:urzSGblok}
	\end{figure}
\end{minipage}
\begin{minipage}{0.05 \textwidth}
	\vspace{-20pt}
	\begin{equation*}
		\hspace{-30pt}
		\left\{\begin{array}{lr}
			+ \\
			0 \hspace{3pt} \vert \\
			- \vert
		\end{array}\right\}
	\end{equation*}
\end{minipage}
\begin{minipage}{0.45 \textwidth}
	Będziemy podpisywać modyfikacje z zablokowanymi  "'$0$ "' oraz  "'$-$ "' jako  "'$+S$ "', analogicznie  "'$0S$ "' oraz  "'$-S$ "'
\end{minipage}


\begin{minipage}{0.6 \textwidth}
	\begin{figure}[H]
		\centering
		\includegraphics[width=0.6\textwidth]{urzsgblok2}
		\caption{Zablokowane atomy poza  "'$+$ "' dwa razy}
		\label{fig:urzSGblok2}
	\end{figure}
\end{minipage}
\begin{minipage}{0.5 \textwidth}
	\vspace{-40pt}
	$\langle +S \vert +S \rangle = 1$, dlatego że wszystkie atomy przejdą najpierw górnym, a potem będą tylko i wyłącznie dostępne atomy z plusem, więc wszystkie atomy znowu przejdą, więc zawsze wychodzi $1$. Jest to postulat.
\end{minipage}


\begin{minipage}{0.6 \textwidth}
	\begin{figure}[H]
		\centering
		\includegraphics[width=0.6\textwidth]{urzsgblok3}
		\caption{Zablokowane atomy raz poza  "'$+$ "' i raz poza  "'$-$ "'}
		\label{fig:urzSGblok3}
	\end{figure}
\end{minipage}
\begin{minipage}{0.5 \textwidth}
	\vspace{-40pt}
	$\langle -S \vert +S \rangle = 0$
\end{minipage}
\newpage
Możemy w taki sposób wymienić różne stany początkowe i końcowe i możemy stworzyć tabelę
\begin{table}[ht]
	\centering
	\begin{tabular}[t]{c|c|c|c}
		& $+S$ & $0S$ & $-S$\\
		\hline
		$+S$ & $1$ & $0$ & $0$ \\
		\hline
		$0S$ & $0$ & $1$ & $0$ \\
		\hline
		$-S$ & $0$ & $0$ & $1$ \\
	\end{tabular}
	\label{tab:stany}
\end{table}

Następnie rozważmy sytuacje gdzie mamy  "'pokręcone "' urządzenie S-G. Niech pierwsze urządzenie będzie się nazywało urządzeniem "'$S$"' a drugie urządzeniem "'$T$"'

\begin{figure}[H]
	\centering
	\includegraphics[width=0.8\textwidth]{SGArot}
	\caption{Obrócone urządzenie S-G, \textit{Źródło: Deepnote}}
	\label{fig:SGArot}
\end{figure}

Jeżeli mamy na początku stan "'$+S$ "'  a następnie przechodzimy przez urządzenie ze stanem "'$-T$ "' to co będziemy mieli? 


\begin{equation*}
	\left\{\begin{array}{lr}
		+ \\
		0 \hspace{3pt} \vert \\
		- \vert
	\end{array}\right\} 
	\left\{\begin{array}{lr}
		+ \vert \\
		0 \hspace{3pt} \vert \\
		- 
	\end{array}\right\}
	\neq 0 \text{,} \hspace{10pt}  \langle-T \vert +S\rangle \neq 0
\end{equation*}
Jeżeli chcemy obliczyć prawdopodobieństwo przejścia ze stanu "'$+S$ "' do stanu "'$+T$ "' to możemy to obliczyć następująco
\begin{equation*}
	\left| \langle\ +T\vert+S \rangle \right|^2 = \langle\ +T\vert+S \rangle \langle\ +T\vert+S \rangle^*
\end{equation*}
Gdy prowadzimy kilka doświadczeń w których będziemy mieli początek w "' $+S$ "' a następnie przejdziemy przez urządzenia "'$+T$ "' i  "'$0T$ "' oraz "'$0T$ "', to prawdopodobieństwo że cząstka przejdzie końcowo przez którekolwiek z urządzeń "'$T$ "' jest jedynką.

\begin{equation*}
	\langle\ +T\vert+S \rangle \langle\ +T\vert+S \rangle^* + \langle\ 0T\vert+S \rangle \langle\ 0T\vert+S \rangle^* +\langle\ -T\vert+S \rangle \langle\ -T\vert+S \rangle^* =1
\end{equation*}
Następnie rozważmy schemat eksperymentu gdzie najpierw mamy urządzenie "' $0S$ "', następnie "' $0T$ "' a końcowo "' $+S$ "'. Wtedy amplituda $\langle\ +S\vert0T \rangle \langle\ 0T\vert0S \rangle^*$ będzie następująca
\begin{equation*}
	\frac{p_2}{p_1} = \frac{\left|\langle\ +S\vert0T \rangle \langle\ 0T\vert0S \rangle\right|^2}{\left|\langle\ 0S\vert0T \rangle \langle\ 0T\vert0S \rangle\right|^2} = \frac{\left|\langle\ +S\vert0T \rangle\right|^2}{\left|\langle\ 0S\vert0T \rangle\right|^2}
\end{equation*}
Rozważmy trzy schematy eksperymentu gdzie $N$ cząstek będzie przechodzić z urządzenia "' $S$ "' do urządzenia "' $T$ "' i znowu do "' $S$ "'.
\begin{equation*}
	\hspace{-50pt} (1)
	\left\{\begin{array}{lr}
		+ \\
		0 \hspace{3pt} \vert \\
		- \vert
	\end{array}\right\} 
	\xrightarrow[N]{}
	\left\{\begin{array}{lr}
		+  \\
		0  \\
		- 
	\end{array}\right\}
	\xrightarrow[N]{}
	\left\{\begin{array}{lr}
		+  \\
		0 \hspace{3pt} \vert \\
		- \vert
	\end{array}\right\}
	\xrightarrow[N]{} \text{To urządzenie nic nie zmienia}
\end{equation*}

\begin{equation*}
	\hspace{-140pt} (2)
	\left\{\begin{array}{lr}
		+ \\
		0 \hspace{3pt} \vert \\
		- \vert
	\end{array}\right\} 
	\xrightarrow[N]{}
	\left\{\begin{array}{lr}
		+  \\
		0  \\
		- 
	\end{array}\right\}
	\xrightarrow[N]{}
	\left\{\begin{array}{lr}
		+ \vert \\
		0  \\
		- \vert
	\end{array}\right\}
	\xrightarrow[0]{} \text{Zero cząstek}
\end{equation*}

\begin{equation*} (3)
	\left\{\begin{array}{lr}
		+ \\
		0 \hspace{3pt} \vert \\
		- \vert
	\end{array}\right\} 
	\xrightarrow[N]{}
	\left\{\begin{array}{lr}
		+ \vert \\
		0 \hspace{3pt} \vert \\
		- 
	\end{array}\right\}
	\xrightarrow[N]{}
	\left\{\begin{array}{lr}
		+ \vert \\
		0  \\
		- \vert
	\end{array}\right\}
	\xrightarrow[\beta\alpha N]{\alpha \leq 1, \beta \leq 1} \text{Tu przechodzi tylko część cząstek}
\end{equation*}
W przypadku eksperymentu $(3)$, urządzenie "' $T$ "' jest bardziej restrykcyjne niż w przypadku eksperymentu $(2)$, jednak mimo tego końcowo przechodzi niezerowa ilość cząstek. To jest przykład zupełnie kwantowy.
Można to zrozumieć analogicznie w kontekście wektorów, przy standardowym rzutowaniu najpierw na oś $OY$ a następnie na $OX$ zawsze kończymy w zerze, jednak gdybyśmy mieli dodatkowe osie $OY'$ oraz $OX'$ i rzutowali po kolei $OY \rightarrow OY' \rightarrow OX$ to pomimo nałożenia dodatkowego restrykcyjnego rzutowania otrzymujemy wynik niezerowy.
Ze schematu drugiego widzimy
\begin{equation*}
	a = 0 = \left|\langle\ 0S \vert +T \rangle \right| \left|\langle\ +T \vert +S \rangle \right| + 
	\left|\langle\ 0S \vert 0T \rangle \right| \left|\langle\ 0T \vert +S \rangle \right| + 
	\left|\langle\ 0S \vert -T \rangle \right| \left|\langle\ -T \vert +S \rangle \right|
\end{equation*}
Amplitudy mamy interferencyjne i w rezultacie mamy 0. Ze schematu pierwszego natomiast
\begin{equation*}
	a = 1 = \left|\langle\ +S \vert +T \rangle \right| \left|\langle\ +T \vert +S \rangle \right| + 
	\left|\langle\ +S \vert 0T \rangle \right| \left|\langle\ 0T \vert +S \rangle \right| + 
	\left|\langle\ +S \vert -T \rangle \right| \left|\langle\ -T \vert +S \rangle \right|
\end{equation*}
\subsection{Bazy}
Niech $\vert +T \rangle = \vert 1 \rangle$, $\vert 0T \rangle = \vert 2 \rangle$, $\vert -T \rangle = \vert 3 \rangle$. Wtedy możemy zapisać amplitudę $a$ jako sumę
\begin{equation*}
	a = \sum_{i = 1}^3 \langle\ +S \vert i \rangle \langle\ i \vert +S \rangle = 1 = \langle\ +S \vert +S \rangle
\end{equation*}
Dzięki temu możemy zapisać niezależnie od tego jakie mamy urządzenia. Dla schematu z urządzeniami $S$ oraz $R$ mamy
\begin{equation*}
	\left\{\begin{array}{lr}
		+ \\
		0 \hspace{3pt} \vert \\
		- \vert
	\end{array}\right\} 
	\left\{\begin{array}{lr}
		+ \\
		0 \hspace{3pt} \vert \\
		- \vert
	\end{array}\right\} 
	\iff a =  \langle\ +R \vert +S \rangle \langle\ = \sum_{i = 1}^3 \langle\ +R \vert i \rangle \langle\ i \vert +S \rangle
\end{equation*}
Zapiszmy $\vert +R \rangle = \vert \chi \rangle$, $\vert +S \rangle = \vert \phi \rangle$, i wtedy mamy zapis które wszędzie można spotkać w mechanice kwantowej
\begin{equation*}
	\langle\ \chi \vert \phi \rangle = \sum_{i} \langle\ \chi \vert i \rangle \langle\ i \vert \phi \rangle
\end{equation*}
Możemy z tego zapisać kilka reguł (pod warunkiem że $i \neq j$)
\begin{equation*}
	\begin{split}
		\langle\ i \vert i \rangle &= 1 \\
		\langle\ i \vert j \rangle &= 1
	\end{split}
\end{equation*}
Przepiszmy równanie które już wcześniej wyliczyliśmy w tym rozdziale.
\begin{equation*}
	\langle\ +T\vert+S \rangle \langle\ +T\vert+S \rangle^* + \langle\ 0T\vert+S \rangle \langle\ 0T\vert+S \rangle^* +\langle\ -T\vert+S \rangle \langle\ -T\vert+S \rangle^* =1.
\end{equation*}
Łącząc je z równaniem dające wynik amplitudy $a = 1$ dostajemy zależności
\begin{equation*}
	\begin{split}
		\langle\ +S \vert +T \rangle &= \langle\ +T \vert +S \rangle^* \\
		\langle\ +S \vert 0T \rangle &= \langle\ 0T \vert +S \rangle^* \\
		\langle\ +S \vert -T \rangle &= \langle\ -T \vert +S \rangle^* 
	\end{split}
\end{equation*}
Daje nam to kolejną regułę
\begin{equation*}
	\langle\ \chi \vert \phi \rangle = \langle\ \phi \vert \chi \rangle^*
\end{equation*}
\subsection{Operatory}
Amplitudę że przejdziemy ze stanu $+S$ do $0R$ możemy zapisać jako
\begin{equation*}
	a = \langle\ 0R \vert \dots \rangle \langle\ \dots \vert \dots \rangle \dots \langle\ \dots \vert \dots \rangle \langle\ \dots \vert +S \rangle  = \langle\ 0R \vert A \vert +S \rangle 
\end{equation*}
gdzie $A$ to są różne operacje związane ze zbiorem urządzeń. Przez $\hat{A}$ będziemy oznaczać operator generalny.
Możemy też zapisać macierz operatora $\hat{A}$. Niech $\vert +S \rangle$, $\vert 0S \rangle$, $\vert -S \rangle$ będą stanami naszej bazy. Wtedy mamy
\begin{table}[ht]
	\centering
	\begin{tabular}[t]{c|c|c|c}
		& $+$ & $0$ & $-$\\
		\hline
		$+$ & $\langle\ + \vert \hat{A} \vert + \rangle $ & $\langle\ + \vert \hat{A} \vert 0 \rangle $ & $\langle\ + \vert \hat{A} \vert - \rangle $ \\
		\hline
		$0$ & $\langle\ 0 \vert \hat{A} \vert + \rangle $ & $\langle\ 0 \vert \hat{A} \vert 0 \rangle $ & $\langle\ 0 \vert \hat{A} \vert - \rangle $ \\
		\hline
		$-$ & $\langle\ - \vert \hat{A} \vert + \rangle $ & $\langle\ - \vert \hat{A} \vert 0 \rangle $ & $\langle\ - \vert \hat{A} \vert - \rangle $ \\
	\end{tabular}
	\label{tab:macierzoperatora}
\end{table}

Powyższe nazywamy macierzą operatora $\hat{A}$ w bazie takiej jak wyżej podanej. Dajemy też czapkę nad operatorem aby zawsze było wiadomo że chodzi nam o operator, ale nie zawsze trzeba to pisać. Operator nigdy nie jest zależny od bazy ale macierz operatora zawsze jest w jakiejś bazie.
Następnie wyprowadźmy parę własności.
\begin{equation*}
	\left\{\begin{array}{lr}
		+ \\
		0  \\
		- 
	\end{array}\right\}_{S} 
	\left\{\begin{array}{lr}
		+ \\
		0 \\
		- 
	\end{array}\right\}_{T_i}
	\hat{A}
	\left\{\begin{array}{lr}
		+ \\
		0  \\
		- 
	\end{array}\right\}_{T_j} 
	\left\{\begin{array}{lr}
		+ \vert \\
		0 \\
		- \vert
	\end{array}\right\}_{R}
	=
	\left\{\begin{array}{lr}
		+ \\
		0  \\
		- 
	\end{array}\right\}_{S} 
	\hat{A}
	\left\{\begin{array}{lr}
		+ \vert \\
		0 \\
		- \vert
	\end{array}\right\}_{R}
\end{equation*}
Możemy powiedzieć że weźmiemy bazę tego urządzenia $T$ i wtedy możemy to zapisać inaczej jako

\begin{equation*}
	\sum_{ij} \langle\ 0R \vert j \rangle \langle\ j \vert A \vert i \rangle \langle\ i \vert +S \rangle = \langle\ OR \vert A \vert +S \rangle
\end{equation*}
Inaczej możemy zapisać, uogólniając tą regułę, że 
\begin{equation*}
	\langle\ \chi \vert A \vert \phi \rangle = \sum_{ij} \langle\ \chi \vert j \rangle \langle\ j \vert A \vert i \rangle \langle\ i \vert \phi \rangle
\end{equation*}
Niech urządzenie $C$ to urządzenie $B$ które stoi tuż po urządzeniu $A$.
\begin{equation*}
	\{C\} = \{A\} \{B\} = \{A\} \left\{\begin{array}{lr}
		+ \\
		0 \\
		- 
	\end{array}\right\}_{T} \{B\}
\end{equation*}
Przez to możemy zapisać
\begin{equation*}
	\langle\ j \vert C \vert i \rangle = \sum_{ij} \langle\ j \vert B \vert k \rangle \langle\ k \vert A \vert i \rangle
\end{equation*}
A także zapisać to jak mnożenie macierzowe
\begin{equation*}
	\hat{C} = \hat{A} \cdot \hat{B} \rightarrow C = A \cdot B
\end{equation*}
\subsection{Przekształcenia bazy}
Powiedzmy że $\hat{A}$ - operator. Mamy też dwie dowolne ortonormalne bazy $\{1\}$ oraz $\{2\}$. Pytanie jest takie: powiedzmy że mamy operator $\langle\ i_1 \vert \hat{A} \vert j_1 \rangle$. W jaki sposób będą wyglądały elementy macierzowe operatora $\hat{A}$ w bazie $\{2\}$? Do tego potrzebujemy przekształcenia bazy.
\begin{equation*}
	\vert i_2 \rangle = \sum_{j_1} D_{i_2 j_1} \vert j_1 \rangle 
\end{equation*}
Wprowadźmy też tutaj nazewnictwo $\langle j \vert$ - stan "bra", $\vert i \rangle$ - stan "kiet".
Jeżeli ${\vert i \rangle}$ - baza, a $\vert \phi \rangle$ - stan, i jeżeli z tych mamy amplitudy $\langle i \vert \phi \rangle = C_i$, $\langle j \vert \phi \rangle = C_j$, to wtedy możemy powiedzieć że
\begin{equation*}
	\vert \phi \rangle = \sum_{i} C_i \vert i \rangle 
\end{equation*}
Gdzie $C_i$ to amplitudy lub współczynniki. Możemy też pisać
\begin{equation*}
	\vert \chi \rangle = \begin{pmatrix} c_1 \\ c_2 \\ c_3 \end{pmatrix} = c_1 \vert 1 \rangle + c_2 \vert 2 \rangle + c_3 \vert 3 \rangle 
\end{equation*}
W takiej sytuacji piszemy że jakieś $\chi$ jest wektorem, jakimś rozłożeniem w bazie przestrzeni którą mamy. 
Niech też wektor "kiet" będzie 
\begin{equation*}
	\vert \chi \rangle = \sum_i \vert i \rangle D_i.
\end{equation*}
Wtedy wektor "bra" możemy zapisać następująco
\begin{equation*}
	\langle \chi \vert = \sum_j \rangle D_i^* \langle i \vert.
\end{equation*}
Dlaczego tak piszemy? Zobaczmy że dzięki naszemu zapisowi mamy poniższe
\begin{equation*}
	\langle \chi \vert \chi \rangle =  1 = \sum_{ij} \langle i \vert j \rangle D_i^* D_j.
\end{equation*}
Wtedy jeżeli mamy policzyć amplitudę $\langle \chi \vert \phi \rangle$ mamy
\begin{equation*}
	\langle \chi \vert \chi \rangle = \sum_{ij} D_j^* \langle j \vert i \rangle C_i = \sum_i D_i^* C_i
\end{equation*}
I w taki sposób będziemy obliczać powyższą amplitudę z przekształceniem bazy. Zapiszmy też
\begin{equation*}
	\langle\ i_2 \vert A \vert j_2 \rangle = \left(\sum_k D_{il}^* \langle l_1 \vert  \right) A \left(\sum_k D_{jk} \vert k_1 \rangle \right) = \sum_{lk} D_{il}^* D_{jk} \langle l_1 \vert A \vert k_1 \rangle
\end{equation*}
Nie wszystkie bazy są takie same, każde zagadnienie potrzebuje odpowiedniej (wygodnej) bazy i będziemy się tego uczyć.
\subsection{Równanie Schrödingera}
Czas wpływa na ewolucje dowolnego układu kwantowego. $U(t_2, t_1)$ - operator ewolucji.
\begin{equation*}
	\langle \chi \vert \hat{U}(t_2, t_1)\vert \phi \rangle - \text{amplituda}, \vert \phi \rangle \xrightarrow{t_1 \rightarrow t_2} \vert \chi \rangle
\end{equation*}
\begin{equation*}
	\langle i \vert \hat{U}(t_2, t_1)\vert j \rangle, \text{ gdzie $t_2$ - czas końcowy, $t_1$ - czas początkowy}
\end{equation*}
Mówimy, że operator ewolucji posiada własność
\begin{equation*}
	U(t_3, t_1) = U(t_3,t_2)U(t_2,t_1) \leftarrow \text{ postulat}
\end{equation*}
Dla $\delta t$ małego czasu możemy wypisywać, że stan układu w czasie
\begin{equation*}
	\vert \psi (t + \Delta t) \rangle = U(t+\Delta t, t\vert \psi(t) \rangle )
\end{equation*}
Możemy obliczyć współczynniki amplitudy dla stanu bazowego i
\begin{equation*}
	C_i(t+\Delta t) \leftarrow \langle i \vert \psi(t + \Delta t) \rangle = \langle i \vert U(t + \Delta t, t) \vert \psi(t)\rangle = \sum_j \langle i \vert U(t + \Delta t, t) \vert j \rangle \langle j \vert \psi(t) \rangle \rightarrow C_j(t)
\end{equation*}
\begin{equation*}
	C_j(t+\Delta t) = \sum_j U_{ij}(t + \Delta t, t) C_j(t)
\end{equation*}
Mamy też postulat
\begin{equation*}
	U_{ij}(t + \Delta t, t) = \delta_{ij} - \frac{i}{\hbar}H_{ij}(t)\Delta t
\end{equation*}
Gdzie $\frac{i}{\hbar}H_{ij}(t)$ jest wygodnym elementem macierzowym i korzystnym później. Gdy mamy operator $U_{ij}(t, t)$ to nic się nie zmienia [trywialne]. Wykonajmy teraz kilka przekształceń
\begin{equation*}
	\begin{split}
		C_j(t + \Delta t) &= \sum_j (\delta_{ij} - \frac{i}{\hbar}H_{ij}(t)\Delta t)C_J(t)\\
		C_j(t + \Delta t) &= C_j(t) - \frac{i}{\hbar}\Delta t \sum_j H_{ij} H(t) C_j(t) \\
		\frac{C_j(t + \Delta t) - C_j(t)}{\Delta t} &= - \frac{i}{\hbar} \sum_j H_{ij}(t) C_j(t)
	\end{split}
\end{equation*}
Korzystając teraz z zapisu $\dot{c}_i(t)$ - pochodna po czasie, zapiszmy
\begin{equation*}
	i\hbar \dot{c}_i(t) = \sum_j H_{ij}(t) C_j(t)
\end{equation*}
Niech także
\begin{equation*}
	\overrightarrow{c}(t) = \begin{pmatrix} c_1(t) \\ c_2(t) \\ \vdots \end{pmatrix}
\end{equation*}
Wtedy możemy zapisać
\begin{equation*}
	i\hbar \overrightarrow{c}(t) =  \hat{H}(t) \hat{C}(t)
\end{equation*}
\textbf{Pytanie} czy skoro H jest zależne od czasu tutaj, to później też takie będzie czy raczej stałe? To będzie zależeć od zagadnienia, ale łatwiej jest jak nie ma tej zależności.
Wiemy też że cząstka musi być w jakimś stanie, więc
\begin{equation*}
	\sum_i \vert C_i(t)\vert^2 = 1.
\end{equation*}

\textbf{Pytanie} skoro c jest niewiadoma to czy H jest znane? H jest znane (sztuka rozsądnego zgadywania) i takie aby równanie było poprawne.

\textbf{Pytanie} czy i to jest stan z bazy? Tak
\subsection{Przykład: Ewolucja atomu z 2 stanami w polu laserowym}
Powiedzmy, że mamy atom z 2 stanami (przybliżenie/uproszczenie). Wtedy mamy
\begin{equation*}
	i\hbar \begin{pmatrix} \dot{c}_1\\ \dot{c}_2 \end{pmatrix} = H \begin{pmatrix} \dot{c}_1\\ \dot{c}_2 \end{pmatrix} , \hspace{5pt} H = \hbar \begin{pmatrix} -\epsilon & \nu \\ \nu & \epsilon  \end{pmatrix}
\end{equation*}
\begin{equation*}
\epsilon, \nu \in \mathcal{R}, \hspace{5pt} c_1(0) = 1, \hspace{5pt} c_2(0) = 0
\end{equation*}
gdzie wartości przy drugim $\hbar$ to operator współdziałania

\textbf{Pytanie} wiemy że $\epsilon$ może byś dowolne, ale kiedy zmienia się $\nu$? Wpływa na to intensywność lasera i częstość lasera. Mając stany bazy i operator współdziałania możemy mieć $\nu$.

\textbf{Pytanie} czy w macierzy H po skosach mają być te same rzeczy? Tak.
Co jeżeli
\begin{equation*}
	H = \hbar \begin{pmatrix} E_1 & 0 \\ 0 & E_2  \end{pmatrix}?
\end{equation*}
\begin{equation*}
	\begin{cases}
		i \hbar \dot{c}_1 = E_1 C_1 \\
		i \hbar \dot{c}_2 = E_2 C_2
	\end{cases}
	\Rightarrow
	\begin{cases}
		C_1 = C_1(0) \exp{\frac{-iE_1t}{\hbar}} \\
		C_2 = \dots
	\end{cases}
\end{equation*}
\subsection{Funkcja falowa}
Elektron może być w stanach x1, x2, $\dots$.
 
\textbf{Pytanie} ile jest stanów, $\infty$ czy nie? Baza może być skończona lub nieskończona. Gdy baza jest nieskończona mamy przestrzeń Hilberta, tak jest w przyrodzie.
\begin{equation*}
	\begin{split}
		\vert \psi \rangle = \sum_i (i \vert X_i) = \int c(x) \vert x \rangle dx \\
		\psi(x) = C(x) - \text{ funkcja falowa} \\
		\langle x \vert \psi \rangle = c_x = c(x)
	\end{split}
\end{equation*}
Często będzie baza własna
\begin{equation*}
	\vert j \rangle = f_j(x) \leftarrow \text{ standardowy sposób opisania stanów i bazy}
\end{equation*}
Współczynniki przy każdym ze stanów, amplitudy, nazywamy funkcją falową. Stan można zapisać jako przekształcenie bazy kombinacją liniową.
\begin{equation*}
	\begin{split}
		\Rightarrow sum \vert c_i \vert^2 = 1 \\
		\Rightarrow \int \vert \psi(x) \vert^2 dx = 1 \\
	\end{split}
\end{equation*}
\begin{equation*}
	\begin{split}
		\langle \phi \vert \psi \rangle &= \sum_i \langle \phi \vert i \rangle \langle i \vert \psi \rangle = \\
		&\int \langle \phi \vert x \rangle \langle x \vert \psi \rangle dx = \\
		& \int \phi^*(x) \psi(x) dx
	\end{split}	
\end{equation*}

\section{Równanie Schrödingera: Własności}

\subsection{Zachowanie prawdopodobieństwa}

W mechanice kwantowej funkcja falowa $\psi(\vec{r}, t)$ opisuje stan cząstki. Z równania Schrödingera

$$
i\hbar \frac{\partial}{\partial t} \psi(\vec{r}, t) = \left( -\frac{\hbar^2}{2m} \nabla^2 + V(\vec{r}, t) \right) \psi(\vec{r}, t),
$$

można wyprowadzić zasady zachowania prawdopodobieństwa. Zakładamy, że potencjał $V(\vec{r}, t)$ jest funkcją ciągłą lub ma skończone skoki (czyli skończoną wysokość skoków, nawet jeśli może być ich nieskończenie wiele). 

Z definicji gęstości prawdopodobieństwa

$$
\rho(\vec{r}, t) = |\psi(\vec{r}, t)|^2,
$$

oraz warunku unormowania funkcji falowej

$$
\int_{\mathbb{R}^3} |\psi(\vec{r}, t)|^2 \, d\vec{r} = 1,
$$

wynika, że całkowite prawdopodobieństwo znalezienia cząstki w całej przestrzeni wynosi 1. Aby to było spełnione w każdym momencie czasu, pochodna całkowitego prawdopodobieństwa po czasie musi być równa zeru:

$$
\frac{d}{dt} \int_V |\psi(\vec{r}, t)|^2 \, d\vec{r} = 0.
$$

Powyższe równanie można przekształcić do postaci równania ciągłości. Najpierw zapisujemy pochodną jako

$$
\int_V \left( \psi^* \frac{\partial \psi}{\partial t} + \frac{\partial \psi^*}{\partial t} \psi \right) \, d\vec{r} = 0.
$$

Podstawiając wyrażenie z równania Schrödingera (i jego sprzężenia zespolonego)

$$
i\hbar \frac{\partial \psi}{\partial t} = \left( -\frac{\hbar^2}{2m} \nabla^2 + V \right) \psi, \quad
-i\hbar \frac{\partial \psi^*}{\partial t} = \left( -\frac{\hbar^2}{2m} \nabla^2 + V \right) \psi^*,
$$

otrzymujemy
\begin{align*}
\int_V \left( \psi^* \frac{\partial \psi}{\partial t} + \frac{\partial \psi^*}{\partial t} \psi \right) \, d\vec{r}
&= \frac{i\hbar}{2m} \int_V \left( \psi^* \nabla^2 \psi - \psi \nabla^2 \psi^* \right) \, d\vec{r} \\
&= \frac{i\hbar}{2m} \int_V \vec{\nabla} \cdot \left( \psi^* \vec{\nabla} \psi - \psi \vec{\nabla} \psi^* \right) \, d\vec{r} \\
&= - \int_V \vec{\nabla} \cdot \vec{j} d\vec{r}.
\end{align*}

Zgodnie z twierdzeniem Gaussa (znanym też jako twierdzenie Greena–Ostrogradskiego), możemy tę całkę objętościową zapisać jako całkę powierzchniową

$$
\int_V \vec{\nabla} \cdot \vec{j} \, d\vec{r} = \int_{\partial V} \vec{j} \cdot d\vec{S},
$$

gdzie

$$
\vec{j}(\vec{r}, t) = \frac{\hbar}{2mi} \left( \psi^* \vec{\nabla} \psi - \psi \vec{\nabla} \psi^* \right) = \Re \left\{ \psi^* \frac{\hbar}{im} \vec{\nabla} \psi \right\}
$$

jest gęstością prądu prawdopodobieństwa. Ostatecznie otrzymujemy równanie ciągłości

$$
\frac{\partial \rho(\vec{r}, t)}{\partial t} + \vec{\nabla} \cdot \vec{j}(\vec{r}, t) = 0,
$$

co formalnie wyraża zasadę zachowania prawdopodobieństwa — analogiczną do równania zachowania masy w hydrodynamice.


\subsection{Obserwable — wielkości obserwowalne}

W mechanice kwantowej \textbf{obserwabla} (ang. observable) to fizyczna wielkość, którą można zmierzyć eksperymentalnie, np. położenie, pęd, energia czy spin. Każdej obserwabli odpowiada hermitowski operator $\hat{A}$ działający na funkcje falowe przestrzeni Hilberta. Wartość średnia obserwabli $\hat{A}$ w stanie opisanym funkcją falową $\psi(\vec{r}, t)$ dana jest przez wyrażenie:

$$
\langle \hat{A} \rangle = \int \psi^*(\vec{r}, t) \hat{A} \psi(\vec{r}, t) \, d\vec{r}.
$$

Operator $\hat{A}$ musi być hermitowski, aby wartości średnie $\langle \hat{A} \rangle$ były liczbami rzeczywistymi, zgodnie z wymaganiami eksperymentu:

$$
\langle \hat{A} \rangle \in \mathbb{R}.
$$

Dla hermitowskiego operatora $\hat{A}$ zachodzi również

$$
\langle \hat{A} \rangle = \int \psi^* (\hat{A} \psi) \, d\vec{r} = \int (\hat{A} \psi)^* \psi \, d\vec{r}.
$$

\paragraph{Przykłady obserwabli:}

\begin{itemize}
    \item Średnia wartość położenia:
    $$
    \langle \hat{\vec{r}} \rangle = \int \psi^*(\vec{r}, t) \vec{r} \psi(\vec{r}, t) \, d\vec{r}.
    $$

    \item Średnia wartość funkcji $f(\vec{r}, t)$:
    $$
    \langle f(\vec{r}, t) \rangle = \int \psi^*(\vec{r}, t) f(\vec{r}, t) \psi(\vec{r}, t) \, d\vec{r}.
    $$

    \item Średnia wartość pędu (w reprezentacji położeniowej, po transformacji Fouriera):
    $$
    \langle \vec{p} \rangle = \int \psi^*(\vec{r}, t) \left( -i\hbar \vec{\nabla} \right) \psi(\vec{r}, t) \, d\vec{r}.
    $$
\end{itemize}

\subsubsection*{Przykład: Operator spinu $\hat{S}$}

Rozważmy cząstkę o trzech możliwych stanach własnych spinu: $\ket{+}$, $\ket{0}$, $\ket{-}$, dla których zachodzi
\begin{align*}
\langle + | \hat{S} | + \rangle &= +1, \\
\langle 0 | \hat{S} | 0 \rangle &= 0, \\
\langle - | \hat{S} | - \rangle &= -1.
\end{align*}

Dla stanu ogólnego
$$
\ket{\psi} = C_+ \ket{+} + C_0 \ket{0} + C_- \ket{-},
$$
średnia wartość operatora spinu wynosi
$$
\langle \hat{S} \rangle = \langle \psi | \hat{S} | \psi \rangle = |C_+|^2 \cdot (+1) + |C_0|^2 \cdot 0 + |C_-|^2 \cdot (-1) = |C_+|^2 - |C_-|^2.
$$

Powyższy wynik pokazuje, że średnia wartość operatora zależy jedynie od składowych stanu własnego spinu o wartościach różniących się znakiem.

\subsection{Twierdzenie Ehrenfesta}

W mechanice klasycznej ruch cząstki opisywany jest przez równania Hamiltona

$$
\begin{cases}
    \dfrac{d \vec{r}}{dt} = \dfrac{\vec{p}}{m}, \\
    \dfrac{d \vec{p}}{dt} = -\vec{\nabla} V(\vec{r}),
\end{cases}
$$

co prowadzi do analogicznych równań dla wartości średnich w mechanice kwantowej

$$
\begin{cases}
    \dfrac{d \langle \vec{r} \rangle}{dt} = \dfrac{\langle \vec{p} \rangle}{m}, \\
    \dfrac{d \langle \vec{p} \rangle}{dt} = -\langle \vec{\nabla} V(\vec{r}) \rangle.
\end{cases}
$$

Dla porównania, druga zasada dynamiki Newtona ma postać:

$$
m \dfrac{d^2 \vec{r}}{dt^2} = -\vec{\nabla} V(\vec{r}) = \vec{F}.
$$

Wartość średnia wektora położenia $\vec{z}$ wyraża się jako

$$
\langle \vec{z} \rangle = 
\begin{pmatrix}
\langle x \rangle \\
\langle y \rangle \\
\langle z \rangle
\end{pmatrix}.
$$

Rozpocznijmy analizę twierdzenia Ehrenfesta od wyznaczenia pochodnej czasowej wartości średniej położenia:
\begin{align*}
\frac{d}{dt} \langle x \rangle &= \frac{d}{dt} \int \psi^*(\vec{r},t) \, x \, \psi(\vec{r},t) \, d\vec{r} = \int \psi^* \, x \, \frac{\partial \psi}{\partial t} \, d\vec{r} + \int \frac{\partial \psi^*}{\partial t} \, x \, \psi \, d\vec{r} \\
&\overset{\text{RS}}{=}\frac{1}{i\hbar} \left( \int \psi^* x \hat{H} \psi \, d\vec{r} - \int (\hat{H} \psi)^* x \psi \, d\vec{r} \right) \\
&= \frac{1}{i\hbar} \left( \int \psi^* x \left( -\frac{\hbar^2}{2m} \nabla^2 \psi + V \psi \right) d\vec{r} - \int \left( -\frac{\hbar^2}{2m} \nabla^2 \psi^* + V \psi^* \right) x \psi \, d\vec{r} \right) \\
&= \frac{i\hbar}{2m} \int \left[ \psi^* x \nabla^2 \psi - (\nabla^2 \psi^*) x \psi \right] d\vec{r} = *.
\end{align*}

Aby uprościć ten wyraz, skorzystamy z tożsamości Greena:
$$
\int_V \left[ u \nabla^2 v + (\vec{\nabla} u) \cdot (\vec{\nabla} v) \right] d\vec{r} = \int_S u (\vec{\nabla} v) \, d\vec{s},
$$

gdzie $u = u(x,y,z)$, $v = v(x,y,z)$ są funkcjami o odpowiednim zachowaniu na brzegu (zanikają do zera). Zatem
\begin{align*}
\int (\nabla^2 \psi^*) x \psi \, d\vec{r} &= \underbrace{\int_S x \psi (\vec{\nabla} \psi^*) \, d\vec{s}}_{=\,0} - \int (\vec{\nabla} \psi^*) \cdot \vec{\nabla}(x \psi) \, d\vec{r} \\
&= - \int (\vec{\nabla} \psi^*) \cdot \vec{\nabla}(x \psi) \, d\vec{r} \\
&= \underbrace{\int_S \psi^* \vec{\nabla}(x \psi) \, d\vec{s}}_{=\,0} + \int \psi^* \nabla^2 (x \psi) \, d\vec{r} \\
&= \int \psi^* \nabla^2 (x \psi) \, d\vec{r}.
\end{align*}

Wracając do wyrażenia $*$, mamy
\begin{align*}
\frac{d}{dt} \langle x \rangle = * &= \frac{i\hbar}{2m} \int \psi^* \left( x \nabla^2 \psi - \nabla^2 (x \psi) \right) d\vec{r} \\
&= -\frac{i\hbar}{m} \int \psi^* \frac{\partial \psi}{\partial x} \, d\vec{r}.
\end{align*}

Z definicji operatora pędu w kierunku $x$:

$$
\hat{p}_x = -i\hbar \frac{\partial}{\partial x},
$$

wynika, że

$$
\frac{d}{dt} \langle x \rangle = \frac{1}{m} \int \psi^* \left(-i\hbar \frac{\partial}{\partial x}\right) \psi \, d\vec{r} = \frac{1}{m} \langle \hat{p}_x \rangle.
$$

Analogicznie, pochodna wartości średniej pędu wyraża się przez

$$
\frac{d}{dt} \langle \hat{p}_x \rangle = - \left\langle \frac{\partial V}{\partial x} \right\rangle.
$$

Powyższe dwa równania stanowią treść twierdzenia Ehrenfesta i pokazują, że średnie wartości położenia i pędu w mechanice kwantowej zmieniają się zgodnie z klasycznymi równaniami ruchu. 

\section{Proste zagadnienia 1D}

\section{Formalizm Mechaniki Kwantowej}
Przed przejściem do właściwego formalizmu mechaniki kwantowej
ustalmy notację. Dla przestrzeni Hilberta $\mathcal H$ iloczyn skalarny
oznaczamy przez $\langle \cdot | \cdot \rangle$. Jest on liniowy ze względu
na drugą zmienną i antylinowy ze względu na pierwszą, czyli wedle konwencji
matematycznej mamy
\begin{equation*}
    \langle x | y \rangle = \langle y, x \rangle.
\end{equation*}
W związku z tym, rozważając przestrzenie Hilberta postaci $L^2(X)$ mamy
\begin{equation*}
    \langle f | g \rangle = \int_X f(x)^* g(x) dx.
\end{equation*}
Sprzeżenie (Hermitowskie) operatora $A$ będziemy oznaczać $A^\dagger$
zamiast $A^*$. To znaczy $\braket{f|Ag} = \braket{A^\dagger f| g}$.
Przypomnijmy kilka własności sprzężenia:
\begin{equation*}
	\begin{split}
		&(cA)^{\dagger} = c^* A^{\dagger}, \\
		&(A + B)^{\dagger} = A^{\dagger} + B^{\dagger},\\
		&(AB)^{\dagger} = B^{\dagger}A^{\dagger}.
	\end{split}
\end{equation*}
Przypomnijmy też, że operator samosprzężony (Hermitowski) to taki, dla którego $A^\dagger = A$.

Poniżej, znajduje się formalne matematyczne wyjaśnienie notacji bra--ket.

Z twierdzenia Riesza, $x \mapsto \langle x | \cdot \rangle$ jest antyliniową
izometrią $\mathcal H \to \mathcal H^*$.
Ciągłe funkcjonały linowe na $\mathcal H$ będziemy oznaczać $\langle x|$,
gdzie $\langle x| (y) = \langle x | y \rangle$.
Po dwukrotnym zastosowaniu twierdzenia Riesza jasnym jest, że $\mathcal H^{**}$
jest liniowo izometryczna z $\mathcal H$ przez $y \mapsto | y \rangle$, gdzie
$|y\rangle (\langle x |) = \langle x | y \rangle$.
Od tej pory będziemy utożsamiać $x$ z $|x\rangle$; w takim wypadku
\begin{equation*}
    |x \rangle \mapsto \langle x |
\end{equation*}
jest antyliniową izometrią występującą w twierdzeniu Riesza.
W przy takim utożsamieniu, dla dowolnego liniowego operatora
$A \colon \mathcal H \to \mathcal H$ mamy (pomijając nawiasy)
$\braket{f|A|g} = \braket{f|Ag} = \braket{A^\dagger f | g}$.

Jest jasnym, że zachodzą następujące własności: dla
$f, g, h \in \mathcal H, \alpha \in \mathbb C$ oraz operatorów
liniowych $A, B, C\colon \mathcal H \to \mathcal H$
\begin{itemize}
    \item $\braket{f|g} = \braket{g|f}^*$,
    \item $\braket{f|\alpha g + h} = \alpha \braket{f|g} + \braket{f|h}$,
    \item $\braket{\alpha f + g|h} = \alpha^* \braket{f|h} + \braket{g|h}$,
    \item $\braket{f|ABC|g} = \braket{A^\dagger f| B | C g}$,
    \item $\langle \phi |  = \langle x | A^{\dagger} \iff | \phi \rangle = A | x \rangle$,
\end{itemize}
etc.
W fizyce często używa się operatorów nieograniczonych, określonych nie na
całym $\mathcal H$ a na pewnej gęstej podprzestrzeni; wówczas powyższe
dalej jest prawdziwe, przy odpowiednim uwzględnieniu dziedzin.




\subsection{Stan układu}
Będziemy mieli w temacie mechaniki kwantowej kilka postulatów. Nie są one tym samym co aksjomaty w matematyce, ale mimo wszystko są czymś, czego nie możemy wyprowadzić, a co jest nam bardzo potrzebne, aby zbudować jakąś teorię.

\textbf{Postulat 1:} Do zespołu układów fizycznych w pewnych przypadkach można przypisać funkcję falową lub funkcję stanu, która zawiera wszystkie informacje, jakie można znać o~tym zespole. Funkcja ta jest zespolona, można pomnożyć tę funkcję przez dowolną liczbę zespoloną (poza zerem) bez zmiany jej znaczenia fizycznego.

To znaczy, że będziemy zajmować się funkcjami
$\Psi \colon [0, t_{\text{max}}) \times \RR^n \to \CC$ gdzie $\Psi(t, \cdot ) \in L^2(\RR^n)$
(krzywa w przestrzeni $L^2(\RR^n)$ --- przy pewnych dodatkowych założeniach będzie
to funkcja falowa)
lub $\psi \colon \RR^n \to \CC$, $\psi \in L^2(\RR^n)$ (punkt w tejże przestrzeni, t.j. stan).
Dla prostych zagadnień $n=1$, dla przestrzeni trójwymiarowej $n=3$, dla
układu $k$ cząstek w przestrzeni trójwymiarowej $n=3k$ itd.

\textbf{Pytanie z sali:} Liczbę zespoloną o module różnym od zera, tak? \textbf{Odp.} Normalizacja ucierpi wtedy, to znaczy będzie inna normalizacja, ale jeżeli wrzucimy do równania Schrödingera, to wynik będzie taki sam. I nawet potem, jak zobaczymy i obliczamy dowolne obserwable, to wtedy niezależnie od normalizacji funkcji otrzymamy ten sam wynik.

Postulat 1.2 jest implikacją postulatu 1.

\textbf{Postulat 1.2:}
\begin{equation*}
	\begin{split}
		I &= \int \vert \psi(\vec{r}, t)\vert^2 d\vec{r} = (1?) \quad \text{ jest skończona} \\
		I &= \int \vert \psi (\vec{r_1}, \dots, \vec{r}_n, t)\vert^2 d\vec{r_1}\dots d\vec{r}_n = (1?) = \int P(\vec{r_1}, \dots, \vec{r}_n, t)d\vec{r_1}\dots d\vec{r}_n
	\end{split}
\end{equation*}

Jedynka jest ze znakiem zapytania, ponieważ tak zazwyczaj wychodzi dla wygody, ale to nie jest tak zawsze.

\iffalse % to jest nieaktualne po uściśleniu postulatu 1
\textbf{Komentarz z sali:} Implikacja mi się wydaje, że może być tylko dlatego, bo chcemy mieć dowolną informację, a to daje nam informację o prawdopodobieństwie położenia, więc jakby to było nieskończone, to byśmy tracili jakąś informację. To jedyne, co mogę wymyślić, dlaczego to jest implikacja, a nie postulat 1.2. \textbf{Odp.} No racja, faktycznie można tak powiedzieć.
\fi

Jeszcze taka rzecz, której nie mieliśmy. Jeżeli chcemy znaleźć prawdopodobieństwo tego, że cząstka znajduje się w jakimś punkcie $r_1$ w jakimś czasie $t$, wtedy możemy wziąć prawdopodobieństwo dla całego układu $n$ cząstek i zcałkować wszystko po współrzędnych zaczynając od $r_2$, czyli po wszystkich cząstkach poza tą, która nas ciekawi:

\begin{equation*}
	P(\vec{r}_1, t) = \int P (\vec{r}_1, \dots, \vec{r}_n, t) d\vec{r}_2\dots d\vec{r}_n
\end{equation*}

To jest rzecz, której nie wprowadziliśmy wcześniej, bo nie mieliśmy układów wielocząstkowych.

Mówiliśmy o tym, że funkcja falowa z jakimś zdefiniowanym pędem to fala płaska, która nie jest całkowalna kwadratowo, to znaczy całka nie daje nam skończonej liczby. Czemu możemy powiedzieć, że nie mamy z tym problemu? \textbf{Z sali:} bo wszechświat jest skończony. \textbf{Odp.} Faktycznie, ma pan rację, bo kiedy mówimy o jakiejś funkcji falowej, to znaczy coś, co chociażby teoretycznie możemy użyć, aby zmierzyć cząstkę. Jeżeli pewna cząstka, na przykład elektron, ma doskonale zdefiniowany pęd, to z nieoznaczoności Heisenberga będzie wynikało, że nieoznaczoność położenia będzie nieskończona. To znaczy, że jeżeli ktoś chce zmierzyć taki elektron, to to oznacza, że on teoretycznie może znaleźć się nie tylko w państwa laboratorium, a również na Marsie, w innej galaktyce i tak dalej. Więc nawet teoretycznie nie możemy sobie wyobrazić takiej sytuacji, gdzie ktoś zmierzy absolutnie dokładnie pęd elektronu. W takim razie będziemy pracować z pakietami falowymi, które są dobrze całkowalne.

\textbf{Postulat 2: Zasada superpozycji dla $\psi$:} Jeżeli mamy dwie $\psi_1, \psi_2$, to spokojnie możemy je ze sobą dodać z jakimiś współczynnikami $c_1, c_2$ i otrzymamy inną funkcję falową. 

\begin{equation*}
	\psi = c_1 \psi_1 + c_2 \psi_2
\end{equation*}

Współczynniki przy funkcjach falowych zawsze są w kontekście liczb zespolonych, są tylko pewne specjalne zagadnienia, kiedy funkcje falowe są rzeczywiste.

\textbf{Funkcja falowa w przestrzeni pędu}
% TODO
\begin{equation*}
	\phi (\vec{p}_1, \dots, \vec{p}_{\omega}, t) = \int |\psi|^2 d \vec{p} = 1
\end{equation*}
\begin{equation*}
	\begin{split}
		\phi (\vec{p}_1, \dots, \vec{p}_{\omega}, t) \equiv (2 \pi \hbar)^{\frac{-3\omega}{2}} \int \text{exp}&\left[ -\frac{i}{\hbar} (\vec{p}_1 \cdot \vec{z}_1 + \dots + \vec{p}_N \cdot \vec{z}_N)  \right] \\
		&\cdot \psi(\vec{z}_1, \dots, \vec{z}_N, t) \hspace{2pt} \text{d}\vec{z}_1 \dots \text{d}\vec{z}_N
	\end{split}
\end{equation*}
\begin{equation*}
	\left< \psi_1 \vert \psi_2 \right> \equiv \int \psi_1^*(\vec{z}) \psi_2(\vec{z})\text{d}\vec{z} 	
\end{equation*}
\subsection{Zmienne dynamiczne a operatory}
\textbf{Postulat 3:} Z każdą zmienną fizyczną powiązany jest operator liniowy. 
Więc kiedy mówimy o~czymś co możemy wymyślić w przypadku fizyki klasycznej (ale nie tylko), na przykład położenie, pęd, energia, wtedy mamy operator który odpowiada tej zmiennej dynamicznej. W mechanice kwantowej będą rzeczy których może nie być w fizyce klasycznej, na przykład spin. 
\iffalse % oczywiste dla każdego po pierwszym roku matematyki
Co to znaczy że operator jest liniowy? Operator taki musi spełniać poniższe
\begin{equation*}
	A(c_1 \psi_1 + c_2 \psi_2) = c_1 A(\psi_1) + c_2 A(\psi_2)
\end{equation*}
\fi
W jaki sposób te operatory zapisujemy? Taki operator może być zależny od różnych parametrów, na przykład Hamiltonian był zależny od położenia i pędu, więc generalnie operatory możemy jako zależny od wszystkich położeń oraz pędów w czasie. Jaki to jest problem? Problem jest taki że my pracujemy albo w przestrzeni położenia albo w przestrzeni pędów, więc oba na raz wyglądają nieco dziwnie, więc możemy zapisać jak poniżej za pomocą operatora Nabla.
\begin{equation*}
	\begin{split}
		&A(\vec{z}_1, \dots, \vec{z}_N, \vec{p}_1, \dots, \vec{p}_N, t) = \\
		&A(\vec{z}_1, \dots, \vec{z}_N, -i \hbar \vec{\nabla}_{z_1}, \dots, t) = \\
		&A(-i \hbar \vec{\nabla}_{p_1}, \dots, p_1, \dots, t)
	\end{split}	
\end{equation*}

\iffalse % oczywiste dla każdego po pierwszym roku matematyki
\textbf{Definicja:} ${a_n \psi}_n$ to jest zbiór wartości własnych a również stanów operatora A $\iff$ $A \psi_n = a_n \psi_n$.
\fi

\textbf{Postulat 4:} Jedynym wynikiem pomiaru zmiennej $A$ jest jedna
z wartości własnych operatora liniowego $A$ skojarzonego z tą zmienną. \\
Ponieważ wszystkie wyniki pomiarów w fizyce są rzeczywiste, zakładamy
że każdy operator odpowiadający mierzalnej wielkości fizycznej
(czyli obserwabli) jest Hermitowski (samosprzężony), t.j. $A^\dagger = A$;
takie operatory mają właśnie rzeczywiste wartości własne.

Może się zdarzyć sytuacja w której mamy jakiś stan p który jest jak poniżej
\begin{equation*}
	|p_x \rangle = e^{-ip_x x}
\end{equation*}
Czy to jest operator? Nie, to nie jest operator, to jest oznaczenie stanu.
Trzeba wyćwiczyć co to jest operator a co to jest wartość własna operatora.

\iffalse $ napisane wyżej
\textbf{Definicja:} Operator Hermitowski A to jest taki operator który dla którego zachodzi
\begin{equation*}
	\left< x | (A\psi) \right> = \langle (Ax) \vert \psi \rangle
\end{equation*}
\begin{equation*}
	\left< x | (A\psi) \right> \equiv \langle x | A \vert \psi \rangle
\end{equation*}
Korzystając z powyższych możemy zapisać
\fi

\iffalse to powinno być znane, jest też wspomniane wyżej.
Jeśli $\psi_n$ są funkcjami własnymi operatora $A$ o wartościach własnych $a_n$, to
\begin{align*}
	&\left.
	\begin{aligned}
		\left< \psi_n | A | \psi_n \right> &= a_n \left< \psi_n | \psi_n \right> \\
		(A \psi_n)^* = a_n^* \psi_n^* &= \langle (A \psi_n) | \psi_n \rangle = a_n^* \left< \psi_n | \psi_n \right>
	\end{aligned}
	\right\}
	\Rightarrow a_n = a_n^*, \quad a_n \in \Re
\end{align*}
\fi

\textbf{Postulat 5:} Jeżeli seria pomiarów zmiennej dynamicznej A zostanie wykonana na zespole układów opisaną funkcją falową $\psi$ wtedy wartość oczekiwana (średnia) tej zmiennej wynosi
\begin{equation*}
	A = \frac{\langle \psi | A | \psi \rangle}{\langle\psi | \psi\rangle}.
\end{equation*}
Tutaj trzeba zwrócić uwagę na jedną bardzo ważną rzecz, mianowicie wszystkie układy tego pomiaru (np. atomy) muszą być w dokładnie takim samym stanie
(opisanym przez $\psi$). Tu jest różnica między statystyką klasyczną
(jak na przykład przy pomiaru prędkości atomów w gazie) a~kwantową
(gdzie wszystkie atomy s.ą przygotowane w dokładnie takim samym
stanie ale proces kiedy mierzymy ten stan jest procesem z jakimś pewnym prawdopodobieństwem).

\textbf{Pytanie z sali:} A dlaczego wprowadzamy formalizm dopiero teraz? \textbf{Odp.} Bo gdyby wprowadzić formalizm mechaniki kwantowej na samym początku to by nie było wiadomo co się dzieje, a tak zaczynaliśmy od prostszych zagadnień i teraz mamy podstawę do tego by to zrozumieć.

% to jest bardzo nieformalna definicja. Studenci matematyki powinni znać poprawną wersję z wcześniejszych kursów.
\iffalse
\textbf{Definicja:} Operator sprzężony $A^{\dagger}$ definiujemy następująco
\begin{equation*}
	\langle x | A^{\dagger} | \psi \rangle = \langle (Ax)  | \psi \rangle = \langle \psi | A | x \rangle^{*}
\end{equation*}
\fi

Możemy też rozpisywać funkcje operatorowe.
\begin{equation*}
	f(z) = \sum_{i = 0}^{\infty} c_i z^i \rightarrow f(A) = \sum_{i = 0}^{\infty} c_i A^i,
\end{equation*}
gdzie $A^i$ oznacza że operator A powtarza swoje działanie $i$-razy, $A^i = \underbrace{A \cdot A \cdot A \cdot \dotsc \cdot A}_{\text{i razy}}$.
W ten sposób możemy zapisać
\begin{equation*}
	\begin{split}
		A^i \psi_n = A \cdot \dotsc \cdot A \psi_n = (a_n)^i \psi_n.
	\end{split}
\end{equation*}
Z powyższych $(*)$ możemy w takim razie połączyć, że
\begin{equation*}
	f(A) \psi_n = f(a_n) \psi_n
\end{equation*}
Ostatnie co możemy z tej definicji wyciągamy to
\begin{equation*}
	[f(A)]^{\dagger} = \sum_{i = 0}^{\infty} c_i^* (A^i)^{\dagger} = f^*(A^{\dagger})
\end{equation*}

Przypomnijmy definicje operatora odwrotnego i operatora unitarnego.
\textbf{Definicja:} B jest odwrotny do operatora A $\iff$ $BA = AB = I$ i zapisujemy $B = A^{-1}$

\textbf{Definicja:} U jest unitarny $\iff$ $U^{-1} = U^{\dagger}$, czyli
$U U^{\dagger} = U^{\dagger} U = I$. 

% Nie znam takiego twierdzenia jak niżej. I nie, tu nie ma napisanego jego dowodu.
Zawsze też możemy zapisać $U = e^{iA}$, gdzie $A$ --- operator Hermitowski,
ponieważ \newline $U^{\dagger} = (e^{iA})^{\dagger} = e^{-iA} = U^{-1}$.

\textbf{Definicja:} A jest operatorem idempotentnym $\iff A^2 = A$

\textbf{Definicja:} Operator $\Lambda$ nazywamy projekcją,
jeśli jest Hermitowski i idempotentny.
Z tego możemy powiedzieć że 
% ciężko stwierdzić, co ma oznaczać napis poniżej bez dodatkowych informacji.
% Być może chodzi o rozkład wszystkich wektorów \psi na x + \phi -- ortogonalne,
% takie że \Lambda x + \phi = \phi --- to znaczy x \in \ker \Lambda, \phi \in \im \Lambda.
\begin{equation*}
	\begin{split}
		&\forall \psi \exists \phi, x \quad \langle \phi | x \rangle = 0, \psi = \phi + x \\
		&\therefore \phi = \Lambda \psi, x = (1 - \Lambda) \psi  \\
		&\Rightarrow \langle \phi | x \rangle = \langle \Lambda \psi | (1 - \Lambda) \psi \rangle =  \langle  \psi | \Lambda - \Lambda^2 | \psi \rangle = 0 
	\end{split}
\end{equation*}
\subsection{Rozłożenie w funkcje własne}
\begin{align*}
	&\left.
	\begin{aligned}
		\therefore \phi_n = \phi_m, \quad n \neq m \Rightarrow A \psi_j = a_j \psi_j \\
		A \psi_i = a_i \psi_i
	\end{aligned}
	\right\}
	\Rightarrow (a_i - a_j)\langle \psi_i | \psi_j \rangle 
\end{align*}
\begin{equation}
	(a_i - a_j)\langle \psi_i | \psi_j \rangle  = \langle a_i \psi_i | \psi_j \rangle - \langle \psi_i | a_j \psi_j \rangle = \langle (A \psi_i) | \psi_j \rangle - \langle \psi_i | (A \psi_j) \rangle = 0 \Rightarrow \langle \psi_i | \psi_j \rangle = \delta_{ij}
\end{equation}
Powyższe pozwala udowodnić że w przypadku gdy stany własne mają różne energie własne to wtedy te stany są ortogonalne.

Stany zdegenorwane $\Rightarrow A\psi_{n^r} = a_r \psi_{n^r}, \quad r = 1, 2, \dots, \alpha \Rightarrow G.S. $

\textbf{Postulat 5:} Funkcję falową reprezentującą dowolny stan dynamiczny można wyrazić jako kombinacje liniową funkcji własnych operatora A, gdzie A jest operatorem związanym z naszą zmienną. Inaczej, $\psi = \sum_n e_n \psi_n$. Liczba stanów nie musi być skończona ale liczba ta nie jest dla nas jakoś specjalnie ciekawa. Będziemy odcinać stany które nie są praktyczne w obliczeniu

Jaka jest różnica między stanem a funkcją falową? \textbf{Odpowiedź z sali:} stan to nasze zaobserwowanie cząstki a funkcja falowa to wszystkie stany jakie ta cząstka może przyjąć. Możemy zapisywać $\langle x | \Xi \rangle \equiv \psi (\vec{x})$, gdzie po lewej mamy stan a po prawej funkcję falową, czyli z definicji zawiera wszystkie informacje aby określić stan. Możemy tez dla pędu zapisać $\langle p | \Xi \rangle \equiv \phi (\vec{p})$. Możemy rozpisywać $$ \psi(t, x_1, x_2) = \psi_1(t, x_1) \psi_2(t, x_2) $$ dla cząstek.
Mówimy że jeżeli w taki sposób jesteśmy w stanie zapisać takie wyrażenie to zbiór $\{\psi_n\}$ jest pełny. Ale nie każdy operator Hermitowski generuje zbiór funkcji własnych. Powiedzmy że jakiś operator ma trzy funkcje własne, wtedy nigdy nie potrafimy rozpisać dowolnej funkcji falowej (na przykład wodoru) jako taką kombinację z tych trzech funkcji, bo to by oznaczało że chcemy nieskończenie wymiarowy wektor i zmieniamy bazę z przestrzeni nieskończenie-wielowymiarowej do bazy trzy-wymiarowej co nie do końca ma sens. Taka baza musi być przynajmniej nieskończona żeby to się dało obliczyć.

Dalej, mówimy że 
\begin{equation*}
	\langle X | Y \rangle = \sum \langle X | \psi_n \rangle \langle \psi_n | Y \rangle
\end{equation*}
Możemy też zapisać operator jednostkowy $ \psi=\sum C_{n} \psi_{n} H $ ponieważ
\begin{equation*}
	\sum_{n}\left|\psi_{n}\right\rangle\left\langle\psi_{n}\right|=I 
\end{equation*}
Zapis $\ket{\psi}\bra{\phi}$ oznacza jednowymiarowy operator liniowy, to znaczy
\begin{equation*}
    \ket{\psi}\braket{\phi|x} = (\braket{\phi|x})\ket{\psi}.
\end{equation*}
Rodzaj zbieżności podanego szeregu pozostaje zagadką.
% na pewno nie jest to zbieżność w normie operatorowej, jako że wówczas wynik
% musiałby być operatorem zwartym a nie jest -- to identyczność.
Natomiast formalnie możemy przeprowadzić następujący rachunek: dla
stanu $\ket{\phi} = \sum_k c_k \ket{\psi_k}$,
\begin{equation*}
    \left(\sum_{n} |\psi_{n} \rangle \langle \psi_{n} |\right) \ket{\phi} = 
	\sum_{n} |\psi_{n} \rangle \langle \psi_{n} | \sum_{k} c_{k} |\psi_{k} \rangle  =\sum_{n}\left|\psi_{n}\right\rangle c_{n}=\phi,
\end{equation*}
więc sam operator nic nie zmienia.
No a w przypadku gdy obliczamy średnią wartość operatora $A$ to jest 
\begin{equation*}
	\langle A\rangle=\langle\psi| A|\psi\rangle= \dotsc = \sum_{n}\left|c_{n}\right|^{2} a_{n}
\end{equation*}
Przy normalizacji $\langle \psi | \psi \rangle = 1$ mamy $\sum |c_n^2| = 1$.
W~przypadku gdy istnieją stany zdegenerowane, $\psi$ możemy rozpisać jako 
\begin{equation*}
	\psi = \sum_n \sum_{r=1}^{\alpha} c_{n_r} \psi_{n_r}
\end{equation*}

Nie zawsze mamy wartości własne dyskretne i ogólnie musimy rozważyć spektrum ciągłe.
Wtedy możemy zapisać $A \psi_a = a \psi_a$ gdzie $\psi_a$ to funkcja własna tego
operatora a $a$ jest wartością własną która jest ciągła i będziemy w tym wypadku wtedy pisać że iloczyn skarany pomiędzy tymi dwoma stanami dotyczącymi różnych wartości własnych będzie po prostu delta funkcji w której argument jest różnicą miedzy tymi wartościami własnymi $\langle \psi_{a'} | \psi_a  \rangle = \delta(a - a')$.
Wtedy też zmienimy trochę \textbf{Postulat 6} i zapiszemy
\begin{equation*}
	\psi = \sum_n c_n \psi_n + \int c(a) \psi_a \text{d}a
\end{equation*}
I powyższe to istotny sposób jeżeli ktoś chce dobrze opisać wszystkie stany własne atomu helu na przykład to wtedy mogą być stany tak zwane podwójne wzbudzone i wtedy nie da się ich normalnie opisać w iny sposób mimo że to są stany w których mówi że rezonansy są dyskretne.
\subsection{Obserwable \& nieoznaczoności}
\textbf{Definicja:} Komutator operatorów $A, B$ to $[A, B] = AB - BA$.
W przypadku gdy mamy $[A, B] = 0$ to mówimy że $A, B$ są przemienne.

\textbf{Twierdzenie:} Niech $A, B$ będą Hermitowskie.
Wtedy są przemienne przemienne $\iff$ posiadają ten sam zbiór stanów własnych.
Zadanie dla czytającego dowieść to twierdzenie.
Mamy następujące własności komutatora
\begin{equation}
	\begin{aligned}
        &[A, B] = - [B, A] & \text{antysymetria} \\
        &[A, B+C] = [A, B] + [A, C] & \text{(dwu-)liniowość}\\
        &[A, BC] = [A,B]C + B[A,C] & \text{reguła Leibniza}\\
        &[A,[B,C]] + [B,[A,C]] + [C,[A,B]] = 0 & \text{tożsamość Jacobiego.}
	\end{aligned}
\end{equation}
\textbf{Twierdzenie:} Niech $\langle A\rangle=\langle\psi| A|\psi\rangle$ oraz
$\langle B\rangle=\langle\psi| B|\psi\rangle$.
Oznaczamy
\begin{equation*}
    \Delta A \equiv \sqrt{\langle (A - \langle A \rangle)^2\rangle}.
\end{equation*}
Wtedy mamy $$\Delta A \Delta B \geq \frac12 |\langle [A, B] \rangle|.$$
\textbf{Dowód:} $A, B$ --- Hermitowskie.
Oznaczmy $\bar{A} \equiv A-\langle A\rangle,  \bar{B} \equiv B-\langle B\rangle $. 
Możemy zapisać $ (\Delta A)^{2}=\langle\bar{A}^{2}\rangle, 
(\Delta B)^{2}=\langle\bar{B}^{2}\rangle $. 
Z liniowości i faktu, że identyczność komutuje z każdym operatorem mamy
$$ [\bar{A}, \bar{B}]=[A-\langle A\rangle, B-\langle B\rangle]=[A, B].$$ 
Dalej definiujemy operator $C \equiv \bar{A}+i \lambda \bar{B}$.
Operator do niego sprzężony to $C^\dagger \equiv \bar{A}-i \lambda \bar{B}$. 
Jesteśmy ciekawi jaka będzie wartość
$$  \langle C C^{\dagger}\rangle=\langle\psi| C C^{\dagger}|\psi\rangle= \langle C^{\dagger} \psi|C^{\dagger} \psi\rangle \geqslant 0  .$$
Podstawiając definicję tego operatora dostajemy
\begin{equation*}
	\langle C C^{+}\rangle =
        \langle (\bar{A}+ i \lambda \bar{B}) (\bar{A}-i \lambda \bar{B})\rangle
        =\underbrace{\langle \bar A^{2}+\lambda^{2} \bar{B}^{2}-i\lambda[\bar{A}, \bar{B}]\rangle}_{\equiv f(\lambda)}.
\end{equation*}
Stąd
\begin{equation*}
	0 \leq f(\lambda) = \langle\bar{A}^{2}\rangle + \lambda^{2}\langle
        \bar B^{2}\rangle - i\lambda\langle[A, B]\rangle.
\end{equation*}
Ponieważ $(\Delta A)^2$ oraz $(\Delta B)^2$ są większe od zera,
przyjmując $\lambda \in \mathbb R$ mamy
\begin{equation*}
	i \lambda \langle[A, B]\rangle \in \mathbb R \quad \implies
    \quad \text{Re}\langle[A, B]\rangle = 0.
\end{equation*}
Rzeczywista funkcja kwadratowa $f(\lambda)$ ma minimum w punkcie
$\lambda_0  = \frac{i}{2} \frac{\langle[A, B]\rangle}{(\Delta B)^2}$, oraz
\begin{equation*}
	f(\lambda_0)  = (\Delta A)^2  +
        \frac{1}{4} \frac{\langle[A, B]\rangle^2}{(\Delta B)^2}.
\end{equation*}
Stąd
\begin{align*}
		0 \leq f(\lambda_0)  = (\Delta A)^2 + \frac{1}{4} \frac{\langle[A, B]\rangle^2}{(\Delta B)^2}
	\implies (\Delta A)^2 \cdot (\Delta B)^2 \geq - \frac14\langle[A, B] \rangle^2
\end{align*}
Ostatecznie, jako że $\langle [A, B] \rangle$ jest czysto urojone,
mamy $-\langle [A, B]\rangle^2 = |\langle [A, B]\rangle|^2$ i
\begin{equation*}
    \Delta A \Delta B \geq \frac 1 2 | \langle [A, B] \rangle |.
\end{equation*}


\textbf{Przykład} Dla $[x, p_x] = i \hbar$ dostaniemy $\Delta x \Delta p \geq \frac12 \hbar$.
Ktoś może zapytać czy możemy stworzyć stan który będzie posiadał faktycznie tutaj znak równości a nie $\geq$. 
Żeby zyskać odpowiedź musimy znaleźć minimalną nieoznaczoność i do tego musimy wziąć $\lambda = \lambda_0$. 
Powiedzmy że mamy ten przypadek $[x, p_x]$. Mamy $\bar{A} = x - \langle x \rangle$, $\bar{B} = p_x - \langle p_x \rangle$, a $\lambda_0 = \frac{-\hbar}{2(\Delta p_x)^2}$. 
Pamiętajmy tez że $C^\dagger \psi = 0$ a także $(\bar{A} - i \lambda_0 \bar{B})\psi = 0$. Rozpisujemy to i mamy
\begin{equation*}
	\begin{split}
		\left(-i \hbar \frac{d}{dx} - \langle p_x \rangle \right)\psi(x) = \frac{2i(\Delta p_x)^2}{\hbar} (x - \langle x \rangle) \psi(x) \\
		\psi(x) = C \cdot \exp \left(\frac{i}{\hbar}\langle p_x \rangle x\right) \cdot \exp \left( - \frac{(\Delta p_x)^2 (x - \langle x \rangle)^2}{\hbar^2} \right)
	\end{split}
\end{equation*}
To są tak zwane stany koherentne. 
\subsection{Przekształcenia Unitarne}
Niech $A = A^{\dagger}$ i niech $A \psi = X$. Dalej niech
$U U^{\dagger} = I$ (na przykład $U$ unitarny) oraz $\psi' = U \psi, X' = UX$.
Powiedzmy że $A' \psi' = X'$. Wtedy
\begin{equation*}
	A' U \psi = UX = UA\psi \quad \implies \quad  A'U = UA.
\end{equation*}
Operację $A \mapsto A'$ możemy zapisać
formalnie w następujący sposób:
\begin{align*}
		A' = A' U U^{\dagger} &= U A U^{\dagger}, \quad \text{lub}\\ 
		A &= U^{\dagger} A U.
\end{align*}
Własności:
\begin{enumerate}
	\item Jeżeli $A = A^{\dagger}$ to $A' = (A')^{\dagger}$
	\item Jeżeli $  A=\alpha B+\beta C D$ to $A^{\prime}=\alpha B^{\prime}+\beta C' D^{\prime}  $ i jeżeli $[A, B] = \gamma$ to $[A', B'] = \gamma$
	\item $A, A'$ posiadają taką samą bazę stanów własnych
	\item $\langle X | A | \psi \rangle = \langle X' | A' | \psi' \rangle$ 
\end{enumerate}
Następna rzecz to nieskończenie małe przekształcenia unitarne (\textit{infinitesimal}).
Niech $U = I + i \epsilon F$, gdzie $\epsilon \ll 1$, a $F = F^{\dagger}$. Chcemy pokazać że $U$ jest unitarny.
\begin{equation*}
	U^{\dagger} U = (I - i \epsilon F^{\dagger})(I + i \epsilon F) = I
        + i \epsilon\underbrace{(F - F^{\dagger})}_{= 0}
        + \underbrace{\epsilon^2F^{\dagger}F}_{=O(\epsilon^2)}
        \approx I.
\end{equation*}
Analogicznie $U U^{\dagger} \approx I$.
$F$ nazywamy \emph{generatorem} przekształcenia $U$.
Teraz chcemy zobaczyć jak to działa na funkcję falową.
Zapiszmy
$$\psi' \equiv \psi + \delta \psi = U \psi = (I + i \epsilon F)\psi
\quad \implies \quad
\delta \psi = i \epsilon F \psi.$$
W przypadku gdy mamy jakiś operator $A$ też możemy tak zadziałać
\begin{equation*}
	A' = A + \delta A = U A U^{\dagger} = \dotsc = A + i \epsilon [F, A] + O(\epsilon^2) 
    \quad \implies \quad \delta A \approx i \epsilon [F, A].
\end{equation*}
\subsection{Wektory i Macierze}
\begin{align*}
	&\left.
	\begin{aligned}
		X = A \psi \\
		\psi = \sum_n c_n \psi_n
	\end{aligned}
	\right\}
	\Rightarrow X = \sum_m d_m \psi_m
\end{align*}
Gdzie $$d_m = \langle \psi_m | X \rangle = \sum_n \langle \psi_m | A | \psi_n \rangle c_n.$$
Mówimy że $\langle \psi_m | A | \psi_n \rangle = A_{m,n}$ to element macierzowy w jakieś bazie.
Inaczej też przypisujemy $ d_m = \sum_n A_{mn}c_n $
i wtedy możemy wprowadzić to jako $\vec{d} = A \vec{c}$.
W przypadku gdy chcemy obliczyć iloczyn skalarny między stanem $X$ a $\psi$ to wtedy
$$ \langle X | \psi \rangle = \sum_n d_n^* c_n = (\vec{d^{\dagger}} \cdot \vec{c}).$$ 
Jeżeli mamy zbiór stanów ${\psi}_n$ i chcemy przekształcić w zbiór ${\phi_m}$ to wtedy piszemy
\begin{equation*}
	\psi_n = \sum_m U_{mn}\phi_n, \hspace{1cm} \text{gdzie} \,\,\, U_{mn} = \langle \phi_m | \psi_n \rangle.
\end{equation*}
Można z tego udowodnić że gdy $U$ jest unitarny oraz $A$ jest Hermitowski to ślad generatora A to ślad A.
\subsection{Równanie Schrödingera}
\textbf{Postulat 7:} Ewolucję czasową funkcji falowej układu wyznacza zależne od czasu równanie Schrödingera które można zapisać następująco:
\begin{equation*}
	i \hbar \frac{\partial}{\partial t} \psi(t)=H \psi(t)
\end{equation*}
gdzie $H$ to hamiltonian - operator całkowitej energii układu.

Zazwyczaj zapisujemy Hamiltonian następująco:
$$  H=\sum_{i=1}^{N} \hat{\overrightarrow{p}}^{2} \frac{1}{2 m_{i}} + U\left(\vec{r}, \ldots \vec{r_{n}}, t\right)  $$
Operator pędu zapisujemy jako $\hat{\vec{p}} = -i \hbar \vec{\nabla}_i$
Następnie wprowadzamy Operator Ewolucji $U(t, t_0)$ przez następujące
aksjomaty: % , czyli operator reprodukujący
% rozwiązania równania Shr\"odingera. Ma on następujące właściwości:
\begin{itemize}
	\item $\psi(t)=U\left(t, t_{0}\right) \psi\left(t_{0}\right)$
	\item $U(t_0, t_0) = I$
	\item $U(t, t_0) = U(t, t')U(t', t_0)$
	\item $U^{-1}(t, t_0) = U(t_0, t)$
\end{itemize}
Kiedy wstawimy to wszystko w RS to dostaniemy 
\begin{equation*}
	\begin{split}
		i \hbar \frac{\partial}{\partial t} U(t, t_0) \psi(t_0) = H U(t, t_0) \psi(t_0) \\
		U(t, t_0) = I - \frac{i}{\hbar} \int_{t_0}^t HU(t', t_0)dt'	
	\end{split}
\end{equation*}
Teraz chcemy pokazać że operator $U$ jest unitarny.
\begin{equation*}
	\begin{split}
		1=\langle\psi(t_{0}) | \psi(t_0)\rangle &= \langle\psi(t) | \psi(t)\rangle = \\
		&=\langle U(t, t_0) \psi(t_0) | U(t, t_0) \psi(t_0) \rangle = \langle \psi(t_0) | U^{\dagger}(t, t_0) U(t, t_0)| \psi(t_0)  \rangle
	\end{split}
\end{equation*}
Aby końcowe wyrażenie było takie samo jak początkowe musimy wprowadzić że $U U^{\dagger} = I$ więc $U$ jest unitarny.
% nie rozumiem tego dowodu. Jakie są tu założenia? Wygląda jakbyśmy zakładali,
% że równanie 

Dalej chcemy rozważać sytuacje że Hamiltonian jest niezależny od czasu, wtedy $$  U\left(t, t_{0}\right)=\exp \left(-\frac{i}{\hbar} H \cdot(t-t_{0})\right)  $$
$$  \psi(t)=\exp \left(-\frac{i}{\hbar} H \cdot(t-t_{0})\right) \psi(t_0)  $$
Na koniec chcemy rozważyć tak zwane równanie Schrödingera dla operatorów. Powiedzmy że mamy operator Hermitowski i chcemy zobaczyć jak będzie się zmieniać jego wartość średnia w czasie. Mamy wtedy 
\begin{equation*}
	\begin{split}
		\frac{d}{dt} \langle A \rangle &= \frac{d}{d t}\langle\psi| A|\psi\rangle \\
		&= \left\langle\frac{\partial \psi}{\partial t}\left|  A\right| \psi\right\rangle+\left\langle\psi| \frac{\partial A}{\partial t}| \psi\right\rangle+\left\langle\psi\left| A\right|\frac{\partial \psi}{\partial t}\right\rangle \\
		&= (-i \hbar)^{-1} \langle H\psi| A|\psi\rangle + \left\langle\psi| \frac{\partial A}{\partial t}| \psi\right\rangle + (i \hbar)^{-1} \langle \psi| A|H\psi\rangle \\
		&= (i\hbar)^{-1} \langle [A, H] \rangle + \langle \frac{\partial A}{\partial t} \rangle \\
		% &= \frac{d}{dt}A
	\end{split}
\end{equation*}
To jest tak zwane równanie Heisenberga. \textbf{Pytanie z sali:} H jest samo-sprzężony tutaj? \textbf{Odp.} Tak, bo wartości własne mają być rzeczywiste bo jest energia, ale niektórzy twierdzą że aby operator miał rzeczywiste wartości własne może spełniać inne warunki i nie musi być Hermitowski.
\textbf{Ważny przykład} kiedy Hamiltonian nie jest zależny od czasu to piszemy $\delta_x \frac{\partial H}{\partial t} = 0$ a także $$ \frac{d}{dt} \langle H \rangle =\langle (i \hbar)^{-1} [H, H] \rangle = 0$$
I to co tutaj jest zapisane jest kwantowo-mechaniczną odpowiedzią do tego że energia się zachowuje. Gdyby tu zamiast $H$ był inny operator niezależny od czasu to by oznaczało że fizycznie zmienna tego operatora jest wartością która się zachowuje.

\section{Moment pędu}

\subsection{Momentu pędu w mechanice klasycznej}
W mechanice klasycznej moment pędu (inaczej moment kątowy) cząstki względem punktu odniesienia definiujemy jako iloczyn wektorowy położenia $\vec{r}$ i pędu $\vec{p}$
$$
\vec{L} = \vec{r} \times \vec{p}.
$$
\begin{figure}[H]
\centering
\begin{tikzpicture}[scale=1.5]
\begin{scope}
  \draw[->, thick] (0,0) -- (2,0) node[below] {$\vec{r}$};
  \draw[->, thick] (0,0) -- (0,2) node[left] {$\vec{L}$};
  \draw[->, thick] (0,0) -- (1.7,1) node[right] {$\vec{p}$};
  \draw (1,0) arc (0:30:1);
  \node at (0.6,0.15) {$\alpha$};
\end{scope}
\end{tikzpicture}
\caption{Moment pędu jako iloczyn wektorowy $\vec{r}$ i $\vec{p}$.}
\end{figure}

Wartość bezwzględna momentu pędu wynosi
$$
|\vec{L}| = |\vec{r}| \cdot |\vec{p}| \cdot \sin \alpha,
$$
gdzie $\alpha$ to kąt między wektorami $\vec{r}$ i $\vec{p}$. Moment pędu jest więc wielkością wektorową, prostopadłą do płaszczyzny rozpiętej przez $\vec{r}$ i $\vec{p}$.
\begin{figure}[H]
\centering
\begin{tikzpicture}[scale=1.5]
\begin{scope}
  \draw[dashed, thick] (0,0) ellipse (1 and 0.6);
  \draw[->, thick] (0,0) -- (1,0) node[midway, above] {$\vec{r}$};
  \draw[->, thick] (1,0) -- (1,0.6) node[right] {$\vec{p}$};
\end{scope}
\end{tikzpicture}
\caption{Moment pędu jako wektor prostopadły do płaszczyzny rozpiętej przez $\vec{r}$ i $\vec{p}$.}
\end{figure}

\subsection{Zasada zachowania momentu pędu}
W układach \textbf{stacjonarnych}, czyli takich, w których nie zmieniają się parametry układu w czasie, zachodzi zasada zachowania momentu pędu
$$
\frac{d\vec{L}}{dt} = 0 \quad \Rightarrow \quad \vec{L} = \text{const.}
$$
Oznacza to, że moment pędu nie ulega zmianie w czasie, jeśli na układ nie działa moment sił zewnętrznych (np. brak momentu zewnętrznego lub centralna symetria pola).

Zachowanie całkowitego momentu pędu wynika z symetrii układu, a w szczególności z~\textbf{symetrii osiowej}. W języku mechaniki klasycznej mówi się wówczas, że jest to \textbf{całka ruchu}.

\subsection{Moment pędu w mechanice kwantowej}
W mechanice kwantowej moment pędu staje się \textbf{operatorem}. Składowe klasycznego momentu pędu można zapisać w postaci
$$
\begin{aligned}
L_x &= y p_z - z p_y, \\
L_y &= z p_x - x p_z, \\
L_z &= x p_y - y p_x.
\end{aligned}
$$

Po przejściu do przestrzeni operatorów (czyli do formalizmu mechaniki kwantowej), każdej z wielkości przypisujemy odpowiedni operator. Operatory składowych momentu pędu przyjmują postać
$$
\begin{aligned}
\hat{L}_x &= -i\hbar \left( y \frac{\partial}{\partial z} - z \frac{\partial}{\partial y} \right), \\
\hat{L}_y &= -i\hbar \left( z \frac{\partial}{\partial x} - x \frac{\partial}{\partial z} \right), \\
\hat{L}_z &= -i\hbar \left( x \frac{\partial}{\partial y} - y \frac{\partial}{\partial x} \right),
\end{aligned}
$$
gdzie $\hbar$ to zredukowana stała Plancka.

Operator momentu pędu jako wielkość wektorowa zapisuje się ogólnie w postaci
$$
\vec{\hat{L}} = -i\hbar\, \vec{r} \times \vec{\nabla},
$$
gdzie $\vec{r} = (x, y, z)$ jest operatorem położenia, a $\vec{\nabla} = \left( \frac{\partial}{\partial x}, \frac{\partial}{\partial y}, \frac{\partial}{\partial z} \right)$ jest operatorem gradientu.

\subsection{Komutatory operatorów momentu pędu}
$$
\begin{aligned}
[L_x, L_y] &= [y p_z - z p_y, z p_x - x p_z] \\
&= [y p_z, z p_x] - [z p_y, z p_x] - [y p_z, x p_z] + [z p_y, x p_z]
\end{aligned}
$$

$$
[yp_z, zp_x] = yp_z zp_x - zp_x yp_z = y p_x [p_z, z] = -i\hbar y p_x
$$

Poniżej przedstawione są komutatory
$$
\begin{aligned}
  [\hat{L}_x, \hat{L}_y] &= i\hbar(xp_y - yp_x) = i \hbar \hat{L}_z, \\
  [\hat{L}_y, \hat{L}_z] &= i \hbar \hat{L}_x, \\
  [\hat{L}_z, \hat{L}_x] &= i \hbar \hat{L}_y.
\end{aligned}
$$
Z powyższych relacji wynika, że operatory składowych momentu pędu nie komutują ze sobą
$$
[\hat{L}_x, \hat{L}_y] \neq 0,
$$
co oznacza, że nie można jednocześnie znać dokładnych wartości wszystkich trzech składowych momentu pędu.

Gdyby nie operatory to poniższa relacja byłaby sprzeczna
$$
\hat{\vec{L}} \times \hat{\vec{L}} = i\hbar \hat{\vec{L}}.
$$

\subsection{Operator $\hat{L}^2$}
Kwadrat całkowitego momentu pędu zdefiniowany jest jako
$$
\hat{L}^2 = \hat{L}_x^2 + \hat{L}_y^2 + \hat{L}_z^2.
$$
Działa on jako operator skalarowy. Co istotne, komutuje ze wszystkimi składowymi momentu pędu
$$
[\hat{L}^2, \hat{L}_x] = [\hat{L}^2, \hat{L}_y] = [\hat{L}^2, \hat{L}_z] = 0.
$$
Zatem można jednocześnie mierzyć $\hat{L}^2$ i jedną wybraną składową, np. $\hat{L}_z$.

\subsection{Wybór układu współrzędnych -- baza sferyczna}
W przypadku momentu pędu wygodnie jest przejść do współrzędnych sferycznych. Przejście z kartezjańskich do sferycznych opisuje się wzorami:
$$
\begin{aligned}
x &= r \sin \theta \cos \varphi, \\
y &= r \sin \theta \sin \varphi, \\
z &= r \cos \theta.
\end{aligned}
$$
W tej bazie wyrażenia dla operatorów $\hat{L}_z$ i $\hat{L}^2$ przyjmują szczególnie prostą postać, co umożliwia rozwiązywanie równań własnych i wyznaczanie funkcji własnych momentu pędu (sferyczne funkcje harmoniczne).


\subsection{Operator momentu pędu w współrzędnych sferycznych}
W układzie sferycznym wyrażenia dla operatorów momentu pędu upraszczają się i zależą tylko od kątów $\theta$ i $\varphi$. Składowe operatora momentu pędu mają następującą postać:
$$
\begin{aligned}
\hat{L}_x &= -i\hbar \left( \sin\varphi \frac{\partial}{\partial\theta} + \cot\theta \cos\varphi \frac{\partial}{\partial\varphi} \right) \\
\hat{L}_y &= -i\hbar \left( -\cos\varphi \frac{\partial}{\partial\theta} + \cot\theta \sin\varphi \frac{\partial}{\partial\varphi} \right) \\
\hat{L}_z &= -i\hbar \frac{\partial}{\partial\varphi}
\end{aligned}
$$

Kwadrat momentu pędu $\hat{L}^2$ w układzie sferycznym wyraża się jako
$$
\hat{L}^2 = -\hbar^2 \left[ \frac{1}{\sin\theta} \frac{\partial}{\partial\theta} \left( \sin\theta \frac{\partial}{\partial\theta} \right) + \frac{1}{\sin^2\theta} \frac{\partial^2}{\partial\varphi^2} \right]
$$

Jest to tzw. \textbf{operator Laplace’a-Beltramiego} na sferze jednostkowej, który jest kluczowy w~opisie funkcji własnych momentu pędu, czyli sferycznych funkcji harmonicznych.

Nie ma zależności między funkcjami własnymi różnych składowych. Z operatorów $\hat{L}_x, \hat{L}_y, \hat{L}_z$ tylko jedna składowa może być jednocześnie diagonalizowana razem z $\hat{L}^2$. Zatem
$$
[\hat{L}_i, f(\theta, \varphi)] \ne 0 \quad \text{dla } i = x, y.
$$

$$
V_n = \vec{n} \cdot \vec{V} = n_x V_x + n_y V_y + n_z V_z \quad ; \quad \vec{w} = \vec{u} \times \vec{v}, \; \norm{\vec{w}} = \norm{\vec{u}} = \norm{\vec{v}} = 1
$$
Z tego wynika analogiczna algebra komutatorów
$$
[L_u, L_v] = i\hbar L_w \quad ; \quad [L_v, L_w] = i\hbar L_u \quad ; \quad [L_w, L_u] = i\hbar L_v
$$

\subsection{Układ złożony z wielu cząstek}
Załóżmy, że mamy $N$ cząstek. Wtedy całkowity moment pędu układu to suma momentów pędu poszczególnych cząstek
$$
\hat{\vec{L}} = \sum_{i=1}^N \vec{r}_i \times \hat{\vec{p}}_i = \sum_{i=1}^N \hat{\vec{L}}_i.
$$

Każdy z operatorów zachowuje tę samą strukturę komutatorów jak w przypadku jednej cząstki
$$
[\hat{L}_x, \hat{L}_y] = i\hbar \hat{L}_z, \quad \text{itd.}
$$


\section{Atom wodoru}
\subsection{Wstęp}
Przypomnijmy sobie, czym jest atom wodoru. Mamy jeden atom i jeden elektron,
dwa ciała, które ze sobą współdziałają: jakieś cząstki A oraz B.
Pytanie: czy możemy jakoś zredukować to zagadnienie do jedno-cząstkowego?
Odpowiedź brzmi: faktycznie możemy. Najpierw możemy zapisać, że potencjał
współdziałania między tymi cząstkami to
$$
V(\vec{r}_A, \vec{r}_B) = V(\vec{r})
$$
gdzie $\vec{r} = \vec{r}_A - \vec{r}_B$. Hamiltonian będzie wtedy wyglądał następująco
\begin{equation*}
	H =  \frac{p_{A}^{2}}{2 m_{A}}+\frac{p_{B}^{2}}{2 m_{B}} + V(\vec{r} _A- \vec{r}_B).
\end{equation*}
Równanie Schrödingera dla tego układu natomiast będzie wyglądać tak
\begin{equation*}
	i \hbar \frac{\partial}{\partial t} \psi \left(\vec{r}_{A}, \vec{r}_{B}, t\right) = \left[-\frac{\hbar^{2}}{2 m_{A}} \vec{\nabla}_{A}^{2}+V(\vec{r})\right] \psi \left(\vec{r}_{A}, \vec{r}_{B}, t\right)
\end{equation*}
Jak teraz to uprościć? Możemy wprowadzić środek masy
\begin{equation*}
	\vec{R}=\frac{m_{A} \vec{r}_{A}+m_{B} \vec{r}_{B}}{m_{A}+m_{B}}, \quad m = m_A + m_B.
\end{equation*}
Wprowadzamy również tzw. masę zredukowaną
\begin{equation*}
	\mu = \frac{m_A m_B}{m_A + m_B}.
\end{equation*}
W takim razie równanie Schrödingera będzie wyglądało następująco:
\begin{align*}
	&\left.
	\begin{aligned}
		i \hbar\dot{\psi}(\vec{R}, \vec{r}, t)  =  \left[-\frac{\hbar^{2}}{2 m} \nabla_{R}^{2}-\frac{\hbar^{2}}{2 \mu} \nabla_{r}^{2}+V(\vec{r})\right]  \psi(\vec{R}, \vec{r}, t) \\
		\psi(\vec{R}, \vec{r}, t)=\phi(\vec{R}) \psi(\vec{r}) \exp \left(\frac{-[E_m + E]t}{\hbar}\right)
	\end{aligned}
	\right\}
	\xRightarrow[\text{zmiennych}]{\text{Metoda rozdzielnych}}
\end{align*}
\begin{align*}
	\xRightarrow[\text{zmiennych}]{\text{Metoda rozdzielnych}}
	&\left\{
	\begin{aligned}
		&\frac{-\hbar^{2}}{2 m} \nabla_{R}^{2} \phi(R)=E_{CM} \phi(R)\\
		&\left[\frac{-\hbar^{2}}{2 \mu} \nabla_{r}^{2}+V(\vec{r})\right] \psi(\vec{r})=E \psi(\vec{r})
	\end{aligned}
	\right.
\end{align*}
Są dwie klasy zagadnień, dla których to równanie ma ciekawe rozwiązania.
W pierwszym przypadku weźmy oscylator harmoniczny
\begin{equation*}
	V(\vec{r})=\frac{1}{2} k_{1} x^{2}+\frac{1}{2} k_{2} y^{2}+\frac{1}{2} k_{3} z^{2}, \quad \vec{r} = (x, y, z)
\end{equation*}
W przypadku, gdy mamy potencjał centralny, możemy oznaczyć
\begin{equation*}
	V(\vec{r}) = V(|\vec{r}|) = V(r).
\end{equation*}
Będziemy pisać $r$ bez strzałki i to po prostu oznacza długość wektora $r$. To jest tzw. potencjał centralny i w tym przypadku wygodnie jest przejść do współrzędnych sferycznych. Wtedy Hamiltonian będzie wyglądał tak:
\begin{equation*}
	\begin{split}
		H &=\frac{-\hbar^{2}}{2 \mu} \nabla^{2}+V(r) \\
		&= \frac{-\hbar^{2}}{2 \mu} \left[ \frac{1}{r^{2}} \frac{\partial}{\partial r} \left( r^{2} \frac{\partial}{\partial r} \right) + \frac{1}{r^{2} \sin \theta} \frac{\partial}{\partial \theta} \left( \sin \theta \frac{\partial}{\partial \theta} \right) + \frac{1}{r^{2} \sin^{2} \theta} \frac{\partial^{2}}{\partial \phi^{2}} \right] +V(r) \\
		&= -\frac{\hbar^2}{2\mu} \left[ \frac{1}{r^2} \frac{\partial}{\partial r} \left( r^2 \frac{\partial}{\partial r} \right) - \frac{1}{\hbar^2 r^2} \hat{L}^2 \right] + V(r)
	\end{split}
\end{equation*}
Wstawiając to do równania Schrödingera, dostajemy
\begin{equation*}
	\left[ -\frac{\hbar^2}{2\mu} \left( \frac{1}{r^2} \frac{\partial}{\partial r} \left( r^2 \frac{\partial}{\partial r} \right) - \frac{1}{\hbar^2 r^2} \hat{L}^2 \right) + V(r) \right] \psi(\vec{r}) = E \psi(\vec{r})
\end{equation*}
Dalej pamiętamy, że w przypadku, gdy mamy jakąś funkcję współrzędnych, to ta funkcja komutuje z operatorem $\hat{L}_x$, $\hat{L}_y$, $\hat{L}_z$ i tak samo z operatorem $\hat{L}^2$.
\begin{equation*}
	\begin{split}
		[\hat{L}_x, \hat{f}(r)] = 0 \\
		[\hat{L}_y, \hat{f}(r)] = 0 \\
		[\hat{L}_z, \hat{f}(r)] = 0 \\
		[\hat{L}^2, \hat{f}(r)] = 0 
	\end{split}
\end{equation*}
\text{Pytanie z sali:} Jak mamy rozumieć ten zapis? \textbf{Odp.} To jest operator, który jest zależny od $r$. Wprowadzamy teraz założenie, jak będzie wyglądała nasza funkcja. Mówimy, że rozwiązanie poprzedniego równania będzie miało kształt:
\begin{equation*}
	\psi_{Elm} = R_{Elm}(r) \psi_{lm}(\Theta, \Phi)
\end{equation*}
Wrzucamy to wszystko do równania Schrödingera i dostajemy
\begin{align*}
	&\left\{
	\begin{aligned}
		&L^2 \psi_{lm}(\Theta, \Phi) =  l(l+1) \hbar^2 \psi_{lm}(\Theta, \Phi)\\
		&\left[-\frac{\hbar^2}{2\mu} \left(\frac{\text{d}^2}{\text{d}r^2} + \frac2r \frac{\text{d}}{\text{d}r} \right) + \frac{l(l+1)\hbar^2}{2 \mu r^2} + V(r)\right] R_{Elm} = E R_{Elm}
	\end{aligned}
	\right.
\end{align*}
Powiedzmy teraz, że będziemy pisać $U_{EL}(r) = r \cdot R_{EL}(r)$ i wtedy równanie Schrödingera będzie miało postać:
\begin{equation*}
	\begin{gathered}
		-\frac{\hbar^2}{2\mu}\frac{d^2 U_{EL}(r)}{dr^2}+V_{\text{eff}}(r)U_{EL}(r)=EU_{EL}(r) \\
		V_{\text{eff}} = V(r)+\frac{l(l+1)\hbar^2}{2\mu r^2}
	\end{gathered}
\end{equation*}
\subsection{Wodór}
Dla wodoru potencjał będzie wyglądał następująco
\begin{equation*}
	\begin{gathered}
		V(r)=-\frac{z e^{2}}{\left(4 \pi \varepsilon_{0}\right)^2}\\
		V_{\text{eff}}(r) = -\frac{z e^2}{4 \pi \varepsilon_0 r} + \frac{l(l+1)\hbar^2}{2\mu r^2}
	\end{gathered}
\end{equation*}
Gdzie ``$z$'' to ładunek jądrowy (1 dla wodoru,
2 dla pierwszego jony Helu i tak dalej). Wrzucamy to wszystko znowu
do równania Schrödingera ale przed tym, dla ułatwienia zapisu zdefiniujemy zmienne
\begin{equation*}
	\begin{gathered}
		\rho=\left(\frac{-8\mu E}{\hbar^2}\right)^{1/2}r\\
		\lambda = \frac{Z e^2}{(4 \pi \varepsilon_0) \hbar} \left(\frac{-\mu}{2E}\right)^{1/2}
	\end{gathered}
\end{equation*}
Wtedy RS się przekształci w 
\begin{equation*}
	\left[\frac{d^{2}}{d \rho^{2}}-\frac{l(l+1)}{\rho^{2}}+\frac{\lambda}{\rho}-\frac{1}{4}\right]U_{El(\rho)}=0.
\end{equation*}
Chcemy zrozumieć jak będą wyglądały nasze rozwiązania. Podejście będzie analogiczne jak kiedy rozwiązywaliśmy dla oscylatora harmonicznego. Najpierw chcemy zrobić jak będzie wyglądało rozwiązane asymptotyczne gdy $\rho \to \infty$, wtedy 
\begin{equation*}
	\left[\frac{\text{d}^{2}}{\text{d} \rho^{2}}-\frac{1}{4}\right]U_{El(\rho)}=0
\end{equation*}
Równanie to daje nam możliwość zapisania jak będzie wyglądać $U_{El}$ gdy $\rho$ dąży do nieskończoności
\begin{equation*}
	U_{El(\rho)} \xrightarrow[\rho \to \infty]{} \exp(\frac{-\rho}{2})
\end{equation*}
Mając to możemy założyć że
\begin{equation*}
	U_{El(\rho)} = \exp(\frac{-\rho}{2}) \cdot f(\rho)
\end{equation*}
No i teraz możemy to wrzucić do RS i będziemy mieli
$$  \left[\frac{d^{2}}{d \rho^2}-\frac{d}{d \rho}-\frac{l(l+1)}{\rho^{2}}+\frac{\lambda}{\rho}\right] f(\rho)=0.  $$
Dalej stosujemy cechowanie $f(\rho) = \rho^{l+1}g(\rho)$.
Będziemy chcieli zrozumieć jakie są wartości własne operatora Hamiltona
i do tego nie potrzebujemy całkowitego rozwiązania tylko takie rozwiązanie,
które określi warunki na znaczenie tych wartości własnych,
i dlatego znowu rozkładamy funkcje $g(\rho)$ w szereg.
\begin{equation*}
	g(\rho)=\sum_{k=0}^{\infty} C_{k} \rho^{k}, \quad c_0 \neq 0
\end{equation*}
Stosujemy najpierw podstawienie $f(\rho) = \rho^{l+1}g(\rho)$; mamy
\begin{equation*}
	\left[\rho\frac{\text{d}^{2}}{\text{d} \rho^2}+(2l + 2 - \rho)\frac{\text{d}}{\text{d} \rho} + (\lambda - l - 1)\right] g(\rho)=0
\end{equation*}
Jeżeli teraz dodamy założenie o postaci $g(\rho)$ otrzymujemy
\begin{equation*}
	\sum_{k=0}^{\infty}\left[ \left( k(k+1) + (2l+2)(k+1) \right)C_{k+1} + (\lambda - l - 1 -k)C_k \right]\rho^k=0
\end{equation*}
Skoro suma jest zerem to każdy składnik jest zerem więc możemy z tego wyznaczyć
\begin{equation*}
	C_{k+1} = \frac{-(\lambda - l - 1 -k)}{(k+1)(2l+2+k)} C_k
\end{equation*}
Zauważmy, że dla pewnych wartości $\lambda$ ciąg $C_k$ jest od pewnego
momentu stale równy zero, a w pozostałych przypadkach
$\frac{C_{k+1}}{C_k} \approx \frac{1}{k}$ przy $k \to \infty$.
Dla takiej asymptotyki otrzymalibyśmy zbieżność szeregu porównywalną
do zbieżności szeregu definiującego eksponentę (bardzo szybką, nieskońkończony
promień zbieżności). Jednak z bliżej niezindetyfikowanego powodu,
% wcześniej bylo napisane, że to ze względu na brak zbieżności, co jest
% nieprawdą.
będziemy zakładać, że $\lambda$ jest dobrana tak, by $C_k$ zerowały się
od pewnego miejsca. Jeśli tym miejscem jest $r_n$, to
\begin{equation*}
	\begin{gathered}
		r_n+ l +1 - \lambda = 0 \\
		\lambda = r_n + l +1
	\end{gathered}
\end{equation*}
Powyżej $n$ nazywamy główną liczbą kwantową. Tutaj możemy zobaczyć że mamy
% Nic tu z niczego nie wynika na moje oko. Nie wiem co ma znaczyć to zdanie poniżej. FIXME
kilka warunków, że $l \in \mathbb{N}$, z tego wynika że $0 \leq l \leq n - 1$.
Teraz wstawiamy to co mieliśmy 
\begin{equation*}
	\begin{split}
		&\lambda = n = \frac{Z e^2}{(4 \pi \epsilon_0) \hbar} \left(\frac{-\mu}{2E}\right)^{1/2} \\
		&\Rightarrow E_n = \frac{\mu}{2\hbar^2} \left( \frac{ze^2}{4 \pi \epsilon_0} \right)^2 \frac{1}{\hbar^2} \\
		&\Rightarrow \frac{-1}{2}\mu c^2 \frac{(z\alpha)^2}{n^2}, \quad n = 1, 2, 3, \dotsc
	\end{split}
\end{equation*}
Gdzie $\alpha$ to jest stała, tak zwana stała struktury. Widzimy tutaj że Energia zależy od $n$ i nie jest zależna od $l$. Wcześniej pisaliśmy energię w równaniu z $l$, jednak teraz widzimy że wartości własne nie zależą od $l$.

Możemy zapisać że $ l\in(0,1,\ldots,n-1) $, $ m\in(-l,\ldots,l) $ i $n\in(1, 2, \dotsc)$. Biorąc to wszystko pod uwagę możemy narysować poziomy energii
\begin{figure}[H]
	\centering
	\includegraphics[width=0.7\textwidth]{poziomyenergii}
	\caption{Poziomy energii.}
	\label{fig:poziomyenergii}
\end{figure}
\subsection{Stany własne}
Wracamy teraz dla równania po podstawieniu funkcji $g(\rho)$
\begin{equation*}
	\left[\rho\frac{\text{d}^{2}}{\text{d} \rho}+(2l + 2 - \rho)\frac{\text{d}}{\text{d} \rho} + (\lambda - l - 1)\right] g(\rho)\neq0
\end{equation*}
Chcielibyśmy wprowadzić nowe zmienne żeby to przekształcić do równania
\begin{equation*}
	z \frac{d w}{d z^{2}}+(c-z) \frac{d w}{d z}-a w=0 
\end{equation*}
Jest to tak zwane równanie Kummer'a-Laplace'a. Wprowadzamy
\begin{equation*}
	\begin{gathered}
		z \equiv \rho \\
		w \equiv g \\
		a \equiv l + 1 - \lambda \\
		c \equiv 2l + 2
	\end{gathered}
\end{equation*}
To równanie jest nam potrzebne ponieważ znamy rozwiązanie tego równania i są
to funkcje hipergeometryczne. Są one definiowane następująco
\begin{equation*}
	\prescript{}{1}{F}_{1}(a, c, z)=1+\frac{a z}{c\cdot1!}+\frac{a(a+1) z^{2}}{c(c+1) 2!}+\ldots=\sum_{k=0}^{\infty} \frac{(a)_{k} z^{k}}{(c)_{k} k!}
\end{equation*}
Gdzie $(\alpha)_k = \alpha (\alpha+1)\dotsc(\alpha+k-1)$, $(\alpha)_0=1$.
Przy $z \to \infty$ zachodzi
\begin{equation*}
	\prescript{}{1}{F}_{1}(a, c, z) \xrightarrow[z \to \infty]{} \frac{\Gamma(c)}{\Gamma(a)}e^z z^{a-c}.
\end{equation*}
Stąd
\begin{equation*}
	\prescript{}{1}{F}_{1}(l + 1 - n, 2l+2, \rho) \rightarrow \rho^{-l-1-\lambda} e^{\rho}
\end{equation*}
Jaki mamy tu problem? Ta funkcja jest rozbieżna.
% Gdzie jest rozbieżna? Tylko w 0.
Musimy z tym szeregiem popracować i znowu gdzieś go odciąć.
% To jest straszne zdanie. Warto je przeformułować. TODO
Mianowicie musimy znowu wyzerować licznik więc znowu $l + 1 - \lambda = -r_n$ i z tego
\begin{equation*}
	\prescript{}{1}{F}_{1}(l + 1 - n, 2l+2, \rho) = \sum_{n=0}^{n - l -1} \frac{(k+l - n)(k -1+l-n)\dots(l+1-n)}{(k+2l+1)(k-1+2l - n)\dots(1 + 2l - 1)} \frac{\rho^k}{k!}
\end{equation*}
I niby mamy rozwiązanie ale tyle nam nie wystarczy, chcemy wyprowadzić jeszcze to przez wielomiany Laguerre'a które mają następującą definicje
\begin{equation*}
	\begin{split}
		L_q^p(\rho)&= \frac{d^p}{d \rho^p}L_q(\rho) \\
		L_q(\rho) &= e^{\rho} \frac{d^q}{d \rho^q}\left(\rho^q e^{-\rho}\right)
	\end{split}
\end{equation*}
To są wielomiany Laguerre'a i możemy rozpisać związek między funkcjami
hipergeometrycznymi a wielomianem Laguerre'a.
\begin{equation*}
	L_{n+l}^{2l+1}(\rho) = -\frac{[(n+l)!]^2}{(n-l-1)!(2l+1)!} \prescript{}{1}{F}_{1}(l + 1 - n, 2l+2, \rho) 
\end{equation*}
Zapiszmy teraz już końcowe rozwiązanie w postaci funkcji radialnej $R$.
\begin{equation*}
	\begin{gathered}
		R_{n l}(\rho)=N \cdot e^{\frac{-\rho}{2}} \rho^{l} L_{n+l}^{2 k+1}(\rho), \quad \text{gdzie} \\
		\rho=\left(\frac{-8\mu E}{\hbar^2}\right)^{1/2}r \\
		N = \left( \left( \frac{2Z}{n a_{\mu}} \right) \frac{(n - l - 1)!}{2n[(n+1)!]^3} \right)^{\frac{1}{2}} \\
		a_{\mu} = \left(\frac{4 \pi \epsilon_0 \hbar^2}{\mu e^2}\right)
	\end{gathered}
\end{equation*}
Teraz wypiszemy pierwsze trzy funkcje i naszkicujemy.
\begin{equation*}
	\begin{split}
		R_{1,0}(r) &=  2 \left(\frac{Z}{a_{\mu}}\right)^{\frac32} \exp(-\frac{zr}{a_{\mu}}) \\
		R_{2,0}(r) &=  2 \left(\frac{Z}{2a_{\mu}}\right)^{\frac32}\left(1 - \frac{zr}{2a_{\mu}}\right) \exp(-\frac{zr}{a_{\mu}}) \\
		R_{2,0}(r) &=  \frac{1}{\sqrt{3}} \left(\frac{Z}{2a_{\mu}}\right)^{\frac32} \left(\frac{zr}{a_{\mu}}\right) \exp(-\frac{zr}{a_{\mu}})
	\end{split}
\end{equation*}
Od razu możemy pomyśleć o tym jakie będzie średnie położenie cząstki
\begin{figure}[H]
	\centering
	\begin{minipage}[t]{0.47\textwidth}
		\centering
		\includegraphics[width=\linewidth]{R10}
		\caption{Wykres $R_{1, 0}$}
		\label{fig:R10}
	\end{minipage}
	\hspace{0.04\textwidth}
	\begin{minipage}[t]{0.47\textwidth}
		\centering
		\includegraphics[width=\linewidth]{R20}
		\caption{Wykres $R_{2, 0}$}
		\label{fig:R20}
	\end{minipage}
\end{figure}
\begin{figure}[H]
	\centering
	\begin{minipage}[t]{0.47\textwidth}
		\centering
		\includegraphics[width=\linewidth]{R21}
		\caption{Wykres $R_{2, 1}$}
		\label{fig:R21}
	\end{minipage}
	\hspace{0.04\textwidth}
	\begin{minipage}[t]{0.47\textwidth}
		\centering
		\includegraphics[width=\linewidth]{R30}
		\caption{Wykres $R_{3, 0}$}
		\label{fig:R30}
	\end{minipage}
\end{figure}
Te wykresy pokazują że średnie położenie cząstki musi poruszać się ,,w prawo'' gdy zwiększamy $r$.

\section{Bozony i fermiony}
\subsection{Symetria}
\paragraph*{Równanie Schrödingera dla układu wielu cząstek}\mbox{}\\
%
\begin{equation*}
    i \hbar \dot{\Psi} = H \psi
\end{equation*}
%
Z tego wynika: 
%
\begin{equation*}
    i \hbar \dot{\Psi}(q_1, \cdots, q_N, t) = H \Psi(q_1, \cdots, q_N, t)
\end{equation*}
%
gdzie $q_i$ to współrzędne położenia i pędu $i$-tej cząstki, N to liczba cząstek.
%
\\ \\
%
Wprowadzamy operator permutacji $P_{ij}$ dla $i$-tej i $j$-tej cząstki:
%
\begin{equation*}
    P_{ij} \Psi(q_1, \cdots, q_i, \cdots, q_j, \cdots, q_N, t) \equiv \Psi(q_1, \cdots, q_j, \cdots, q_i, \cdots, q_N, t)
\end{equation*}
%
Będziemy rozważać cząstki identyczne, czyli nierozróżnialne. Wtedy:
%
\begin{equation*}
    H(P_{ij}, \Psi) = P_{ij} (H \Psi) => [H, P_{ij}] = 0
\end{equation*}
%
Zatem operator $H$ jest symetryczny względem permutacji $P_{ij}$.
%
Operator $P_{ij}$ jest operatorem liniowym i unitarnym, czyli $P_{ij}^2 = I$. Zatem $P_{ij} = \pm 1$, gdzie $\pm 1$ są wartościami własnymi.
%
\paragraph*{Funkcje falowe}\mbox{}\\
%
Ze względu na własności operatora permutacji, istnieją dwa typy funkcji falowych dla układów identycznych cząstek.
Pierwszy typ - funkcje symetryczne $\psi_+$, które pozostają niezmienione pod działaniem operatora permutacji:
%
\begin{equation*}
    \begin{aligned}
        P_{ij} \psi_+ (q_1, \cdots, q_i, \cdots, q_j, \cdots, q_N, t) &\equiv \psi_+ (q_1, \cdots, q_j, \cdots, q_i, \cdots, q_N, t) \\
        &\equiv \psi_+ (q_1, \cdots, q_i, \cdots, q_j, \cdots, q_N, t)
    \end{aligned}
\end{equation*}
%
Drugi typ - funkcje antysymetryczne $\psi_-$, które zmieniają znak pod działaniem operatora permutacji.
Oznacza to, że zamiana miejscami dwóch identycznych cząstek prowadzi do zmiany znaku całej funkcji falowej:
%
\begin{equation*}
    P_{ij} \Psi_- (q_1, \cdots, q_i, \cdots, q_j, \cdots, q_N, t) = - \Psi_- (q_1, \cdots, q_i, \cdots, q_j, \cdots, q_N, t)
\end{equation*}
Ta antysymetryczna właściwość jest fundamentalna dla fermionów i prowadzi bezpośrednio do zasady wykluczenia Pauliego.
Funkcje antysymetryczne charakteryzują się tym, że prawdopodobieństwo znalezienia dwóch fermionów w tej samej pozycji wynosi zero,
co oznacza, że fermiony "unikają" siebie nawzajem.
%
\\ \\
%
Możemy rozważyć operator uogólniony $\hat{P}: P \Psi = P_{ij}P_{ik} \cdots \Psi$
%
\begin{equation*}
    P \Psi  (q_1, \cdots, q_n) = \Psi(q_{P_1}, \cdots, q_{P_n})
\end{equation*}
%
gdzie $P$ to permutacja i-tej cząstki.
%
\begin{equation*}
    P \psi_+ \equiv \psi_+
\end{equation*}
%
\begin{equation*}
    P \psi_- =
    \begin{cases}
        \psi_- & \text{dla parzystej permutacji} \\
        - \psi_- & \text{dla nieparzystej permutacji}
    \end{cases}
\end{equation*}
%
\paragraph*{Postulat}\mbox{}\\
%
Jeżeli spin $s = \frac{n}{2}$, $n \in \mathbb{N}$ - układ identycznych cząstek znajduje się w antysymetrycznym stanie $\psi_-$ i mamy \textbf{fermiony}.
Jeżeli spin $s = n$, $n \in \mathbb{N}$ - układ identycznych cząstek znajduje się w symetrycznym stanie $\psi_+$ i mamy \textbf{bozony}.
%
\paragraph*{Przykłady}\mbox{}\\
\begin{itemize}
    \item Elektron: $s = \frac{1}{2}$ - fermion
    \item Proton: $s = \frac{1}{2}$ - fermion
    \item atom: może być bozonem lub fermionem
\end{itemize}
%
\subsection{Całkowicie symetryczne oraz całkowicie antysymetryczne \\ funkcje falowe}
%
Możemy oznaczyć funkcje falowe symetryczne i antysymetryczne jako:
%
\begin{equation*}
    \Psi_s \equiv \Psi_+ \text{ oraz } \Psi_a \equiv \Psi_-
\end{equation*}
Rozważamy układ N cząstek.
%
Dla $N = 2$ mamy:
%
\begin{equation*}
    \Psi_s (q_1, q_2) = \frac{1}{\sqrt{2}} \left( \Psi(q_1, q_2) + \Psi(q_2, q_1) \right)
\end{equation*}
%
\begin{equation*}
    \Psi_a (q_1, q_2) = \frac{1}{\sqrt{2}} \left( \Psi(q_1, q_2) - \Psi(q_2, q_1) \right)
\end{equation*}
%
Dla $N = 3$ mamy:
%

\begin{align*}
    \Psi_s (q_1, q_2, q_3) &= \frac{1}{\sqrt{6}} \big( 
        \Psi(q_1, q_2, q_3) + \Psi(q_1, q_3, q_2) + \Psi(q_2, q_1, q_3) \\
        &\quad + \Psi(q_2, q_3, q_1) + \Psi(q_3, q_1, q_2) + \Psi(q_3, q_2, q_1)
    \big)
\end{align*}
%
Jest to suma po wszystkich permutacjach trzech cząstek.
%
Zakładamy brak oddziaływań między cząstkami, tzn. zapisujemy Hamiltonian w postaci:
%
\begin{equation*}
    H(q_1, \cdots, q_N) = h_1(q_1) \cdot h_2(q_2) \cdots h_N(q_N)
\end{equation*}
%
Wtedy możemy rozpisać Równanie Schrödingera:
%
\begin{equation*}
    h_i u_i = \epsilon_i u_i \quad \forall_{i \in (0, N], i \in \mathbb{N}}
\end{equation*}
%
Dla $N = 2$ mamy:
%
\\
%
Dla symetrycznej funkcji falowej:
\begin{equation*}
    \Psi_s(q_1, q_2) = \frac{1}{\sqrt{2}} (u_\alpha(q_1) u_\beta(q_2) + u_\beta(q_1) u_\alpha(q_2))
\end{equation*}
%
Stąd wyznacznik:
\begin{equation*}
    \begin{vmatrix}
        u_\alpha(q_1) & u_\beta(q_1) \\
        u_\alpha(q_2) & u_\beta(q_2)
    \end{vmatrix}
\end{equation*}
%
Dla antysymetrycznej funkcji falowej zmienia się znak:
%
\begin{equation*}
    \Psi_a(q_1, q_2) = \frac{1}{\sqrt{2}} (u_\alpha(q_1) u_\beta(q_2) - u_\beta(q_1) u_\alpha(q_2))
\end{equation*}
%
$\Psi_a$ oraz $\Psi_s$ nie są prostymi iloczynami $u_\alpha$ i $u_\beta$.
Oznacza to, że są to stany splątane kwantowo - cząstki nie mogą być opisane niezależnie od siebie.
Udowodnimy że to "stan splątania. Dla uogólnionego przypadku $N = N$ (N cząstek) mamy:
%
\begin{equation*}
    \Psi_\triangle (q_1, \cdots, q_N) = \frac{1}{\sqrt{N!}}
    \begin{vmatrix}
        u_\alpha(q_1) & u_\beta(q_1) & \cdots & u_\nu(q_1) \\
        u_\alpha(q_2) & u_\beta(q_2) & \cdots & u_\nu(q_2) \\
        \vdots & \vdots & \ddots & \vdots \\
        u_\alpha(q_N) & u_\beta(q_N) & \cdots & u_\nu(q_N)
    \end{vmatrix}
\end{equation*}
%
Wyznacznik, który się pojawił jest wyznacznikiem Slatera.
%
Sumując powyższe wzory otrzymujemy:
%
\begin{equation*}
    \Psi_A (q_1, \cdots, q_N) = \frac{1}{\sqrt{N!}} \sum_{P} (-1)^{P} P u_\alpha(q_1) u_\beta(q_2) \cdots u_\nu(q_N)
\end{equation*}
%
\begin{equation*}
    \Psi_S (q_1, \cdots, q_N) = \frac{1}{\sqrt{N!}} \sum_{P} P u_\alpha(q_1) u_\beta(q_2) \cdots u_\nu(q_N)
\end{equation*}
%
\subsection{Zasada Pauliego}
%
Rozważamy antysymetryczny układ elektronów (fermiony). \textbf{Pytanie}: Co będzie, gdy $u_\alpha = u_\beta$?
%
Gdy $u_\alpha$ = $u_\beta$ to wyznacznik Slatera jest równy 0.
%
\begin{equation*}
    \Psi_A (q_1, \cdots, q_N) = 0
\end{equation*}
%
Zatem dwa elektrony nie mogą znajdować się w tym samym stanie.
%
\textbf{Zasada}: Tylko jeden fermion może okupować dany stan kwantowy.
%
\paragraph*{Przykład}\mbox{}\\
%
$l < n$, $l \in \mathbb{N}$, $-l \leq m \leq l$
%
\\ \\
%
Przykład diagramu energetycznego pokazującego różne stany kwantowe dla elektronów w atomie.
\begin{figure}[H]
    \centering
    \includegraphics[width=0.5\textwidth]{bozony_fermiony}
    \label{fig:bozony_fermiony}
\end{figure}
%
\begin{equation*}
    N_e(1s^2, 2s^2, 2p^6)
\end{equation*}
%
Liczba w wykładniku to liczba elektronów w danym stanie (w danym orbitalu na danym poziomie). Zmienna $n$ numeruje poziomy energetyczne. Na pierwszym
poziomie występuje jeden orbital $1s$, na drugim poziomie występują dwa orbitale $2s$ i $2p$ - mieszczą odpowiednio 2, 2, 6 elektronów. Zmienna $l$ to
orbitalna liczba kwantowa - $0$ odpowiada orbitalowi $s$, $1$ odpowiada orbitalowi $p$, $2$ odpowiada orbitalowi $d$, $3$ odpowiada orbitalowi $f$.
Zmienna $m$ to magnetyczna liczba kwantowa - opisuje orientację orbitalu w przestrzeni. Orbital $s$ ma tylko jedną możliwą orientację,
stąd $m = 0$. Orbital $p$ ma trzy możliwe orientacje, stąd $m = -1, 0, 1$.
\paragraph*{Wniosek}\mbox{}\\
%
Bez tej zasady świat byłby inny, albo by nie istniał. Bez tej zasady wszystkie elektrony byłyby w tym samym stanie (skondensowałyby się w jednym stanie),
czyli w najniższym stanie energetycznym, co oznaczałoby brak różnorodności pierwiastków chemicznych i stabilnych struktur molekularnych.
%
Zasada Pauliego jest podstawową własnością cząstek.
%
\paragraph*{Pytanie}\mbox{}\\
%
\textbf{Pytanie}: Czym są strzałki góra/dół? Strzałka reprezentuje spin elektronu, który może mieć wartość $m_s = \pm \frac{1}{2}$ (magnetyczna liczba spinowa).

\end{document}
