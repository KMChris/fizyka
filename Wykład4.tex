\section{Równanie Schrödingera: Własności}

\subsection{Zachowanie prawdopodobieństwa}

W mechanice kwantowej funkcja falowa $\psi(\vec{r}, t)$ opisuje stan cząstki. Z równania Schrödingera

$$
i\hbar \frac{\partial}{\partial t} \psi(\vec{r}, t) = \left( -\frac{\hbar^2}{2m} \nabla^2 + V(\vec{r}, t) \right) \psi(\vec{r}, t),
$$

można wyprowadzić zasady zachowania prawdopodobieństwa. Zakładamy, że potencjał $V(\vec{r}, t)$ jest funkcją ciągłą lub ma skończone skoki (czyli skończoną wysokość skoków, nawet jeśli może być ich nieskończenie wiele). 

Z definicji gęstości prawdopodobieństwa

$$
\rho(\vec{r}, t) = |\psi(\vec{r}, t)|^2,
$$

oraz warunku unormowania funkcji falowej

$$
\int_{\mathbb{R}^3} |\psi(\vec{r}, t)|^2 \, d\vec{r} = 1,
$$

wynika, że całkowite prawdopodobieństwo znalezienia cząstki w całej przestrzeni wynosi 1. Aby to było spełnione w każdym momencie czasu, pochodna całkowitego prawdopodobieństwa po czasie musi być równa zeru:

$$
\frac{d}{dt} \int_V |\psi(\vec{r}, t)|^2 \, d\vec{r} = 0.
$$

Powyższe równanie można przekształcić do postaci równania ciągłości. Najpierw zapisujemy pochodną jako

$$
\int_V \left( \psi^* \frac{\partial \psi}{\partial t} + \frac{\partial \psi^*}{\partial t} \psi \right) \, d\vec{r} = 0.
$$

Podstawiając wyrażenie z równania Schrödingera (i jego sprzężenia zespolonego)

$$
i\hbar \frac{\partial \psi}{\partial t} = \left( -\frac{\hbar^2}{2m} \nabla^2 + V \right) \psi, \quad
-i\hbar \frac{\partial \psi^*}{\partial t} = \left( -\frac{\hbar^2}{2m} \nabla^2 + V \right) \psi^*,
$$

otrzymujemy
\begin{align*}
\int_V \left( \psi^* \frac{\partial \psi}{\partial t} + \frac{\partial \psi^*}{\partial t} \psi \right) \, d\vec{r}
&= \frac{i\hbar}{2m} \int_V \left( \psi^* \nabla^2 \psi - \psi \nabla^2 \psi^* \right) \, d\vec{r} \\
&= \frac{i\hbar}{2m} \int_V \vec{\nabla} \cdot \left( \psi^* \vec{\nabla} \psi - \psi \vec{\nabla} \psi^* \right) \, d\vec{r} \\
&= - \int_V \vec{\nabla} \cdot \vec{j} d\vec{r}.
\end{align*}

Zgodnie z twierdzeniem Gaussa (znanym też jako twierdzenie Greena–Ostrogradskiego), możemy tę całkę objętościową zapisać jako całkę powierzchniową

$$
\int_V \vec{\nabla} \cdot \vec{j} \, d\vec{r} = \int_{\partial V} \vec{j} \cdot d\vec{S},
$$

gdzie

$$
\vec{j}(\vec{r}, t) = \frac{\hbar}{2mi} \left( \psi^* \vec{\nabla} \psi - \psi \vec{\nabla} \psi^* \right) = \Re \left\{ \psi^* \frac{\hbar}{im} \vec{\nabla} \psi \right\}
$$

jest gęstością prądu prawdopodobieństwa. Ostatecznie otrzymujemy równanie ciągłości

$$
\frac{\partial \rho(\vec{r}, t)}{\partial t} + \vec{\nabla} \cdot \vec{j}(\vec{r}, t) = 0,
$$

co formalnie wyraża zasadę zachowania prawdopodobieństwa — analogiczną do równania zachowania masy w hydrodynamice.


\subsection{Obserwable — wielkości obserwowalne}

W mechanice kwantowej \textbf{obserwabla} (ang. observable) to fizyczna wielkość, którą można zmierzyć eksperymentalnie, np. położenie, pęd, energia czy spin. Każdej obserwabli odpowiada hermitowski operator $\hat{A}$ działający na funkcje falowe przestrzeni Hilberta. Wartość średnia obserwabli $\hat{A}$ w stanie opisanym funkcją falową $\psi(\vec{r}, t)$ dana jest przez wyrażenie:

$$
\langle \hat{A} \rangle = \int \psi^*(\vec{r}, t) \hat{A} \psi(\vec{r}, t) \, d\vec{r}.
$$

Operator $\hat{A}$ musi być hermitowski, aby wartości średnie $\langle \hat{A} \rangle$ były liczbami rzeczywistymi, zgodnie z wymaganiami eksperymentu:

$$
\langle \hat{A} \rangle \in \mathbb{R}.
$$

Dla hermitowskiego operatora $\hat{A}$ zachodzi również

$$
\langle \hat{A} \rangle = \int \psi^* (\hat{A} \psi) \, d\vec{r} = \int (\hat{A} \psi)^* \psi \, d\vec{r}.
$$

\paragraph{Przykłady obserwabli:}

\begin{itemize}
    \item Średnia wartość położenia:
    $$
    \langle \hat{\vec{r}} \rangle = \int \psi^*(\vec{r}, t) \vec{r} \psi(\vec{r}, t) \, d\vec{r}.
    $$

    \item Średnia wartość funkcji $f(\vec{r}, t)$:
    $$
    \langle f(\vec{r}, t) \rangle = \int \psi^*(\vec{r}, t) f(\vec{r}, t) \psi(\vec{r}, t) \, d\vec{r}.
    $$

    \item Średnia wartość pędu (w reprezentacji położeniowej, po transformacji Fouriera):
    $$
    \langle \vec{p} \rangle = \int \psi^*(\vec{r}, t) \left( -i\hbar \vec{\nabla} \right) \psi(\vec{r}, t) \, d\vec{r}.
    $$
\end{itemize}

\subsubsection*{Przykład: Operator spinu $\hat{S}$}

Rozważmy cząstkę o trzech możliwych stanach własnych spinu: $\ket{+}$, $\ket{0}$, $\ket{-}$, dla których zachodzi
\begin{align*}
\langle + | \hat{S} | + \rangle &= +1, \\
\langle 0 | \hat{S} | 0 \rangle &= 0, \\
\langle - | \hat{S} | - \rangle &= -1.
\end{align*}

Dla stanu ogólnego
$$
\ket{\psi} = C_+ \ket{+} + C_0 \ket{0} + C_- \ket{-},
$$
średnia wartość operatora spinu wynosi
$$
\langle \hat{S} \rangle = \langle \psi | \hat{S} | \psi \rangle = |C_+|^2 \cdot (+1) + |C_0|^2 \cdot 0 + |C_-|^2 \cdot (-1) = |C_+|^2 - |C_-|^2.
$$

Powyższy wynik pokazuje, że średnia wartość operatora zależy jedynie od składowych stanu własnego spinu o wartościach różniących się znakiem.

\subsection{Twierdzenie Ehrenfesta}

W mechanice klasycznej ruch cząstki opisywany jest przez równania Hamiltona

$$
\begin{cases}
    \dfrac{d \vec{r}}{dt} = \dfrac{\vec{p}}{m}, \\
    \dfrac{d \vec{p}}{dt} = -\vec{\nabla} V(\vec{r}),
\end{cases}
$$

co prowadzi do analogicznych równań dla wartości średnich w mechanice kwantowej

$$
\begin{cases}
    \dfrac{d \langle \vec{r} \rangle}{dt} = \dfrac{\langle \vec{p} \rangle}{m}, \\
    \dfrac{d \langle \vec{p} \rangle}{dt} = -\langle \vec{\nabla} V(\vec{r}) \rangle.
\end{cases}
$$

Dla porównania, druga zasada dynamiki Newtona ma postać:

$$
m \dfrac{d^2 \vec{r}}{dt^2} = -\vec{\nabla} V(\vec{r}) = \vec{F}.
$$

Wartość średnia wektora położenia $\vec{z}$ wyraża się jako

$$
\langle \vec{z} \rangle = 
\begin{pmatrix}
\langle x \rangle \\
\langle y \rangle \\
\langle z \rangle
\end{pmatrix}.
$$

Rozpocznijmy analizę twierdzenia Ehrenfesta od wyznaczenia pochodnej czasowej wartości średniej położenia:
\begin{align*}
\frac{d}{dt} \langle x \rangle &= \frac{d}{dt} \int \psi^*(\vec{r},t) \, x \, \psi(\vec{r},t) \, d\vec{r} = \int \psi^* \, x \, \frac{\partial \psi}{\partial t} \, d\vec{r} + \int \frac{\partial \psi^*}{\partial t} \, x \, \psi \, d\vec{r} \\
&\overset{\text{RS}}{=}\frac{1}{i\hbar} \left( \int \psi^* x \hat{H} \psi \, d\vec{r} - \int (\hat{H} \psi)^* x \psi \, d\vec{r} \right) \\
&= \frac{1}{i\hbar} \left( \int \psi^* x \left( -\frac{\hbar^2}{2m} \nabla^2 \psi + V \psi \right) d\vec{r} - \int \left( -\frac{\hbar^2}{2m} \nabla^2 \psi^* + V \psi^* \right) x \psi \, d\vec{r} \right) \\
&= \frac{i\hbar}{2m} \int \left[ \psi^* x \nabla^2 \psi - (\nabla^2 \psi^*) x \psi \right] d\vec{r} = *.
\end{align*}

Aby uprościć ten wyraz, skorzystamy z tożsamości Greena:
$$
\int_V \left[ u \nabla^2 v + (\vec{\nabla} u) \cdot (\vec{\nabla} v) \right] d\vec{r} = \int_S u (\vec{\nabla} v) \, d\vec{s},
$$

gdzie $u = u(x,y,z)$, $v = v(x,y,z)$ są funkcjami o odpowiednim zachowaniu na brzegu (zanikają do zera). Zatem
\begin{align*}
\int (\nabla^2 \psi^*) x \psi \, d\vec{r} &= \underbrace{\int_S x \psi (\vec{\nabla} \psi^*) \, d\vec{s}}_{=\,0} - \int (\vec{\nabla} \psi^*) \cdot \vec{\nabla}(x \psi) \, d\vec{r} \\
&= - \int (\vec{\nabla} \psi^*) \cdot \vec{\nabla}(x \psi) \, d\vec{r} \\
&= \underbrace{\int_S \psi^* \vec{\nabla}(x \psi) \, d\vec{s}}_{=\,0} + \int \psi^* \nabla^2 (x \psi) \, d\vec{r} \\
&= \int \psi^* \nabla^2 (x \psi) \, d\vec{r}.
\end{align*}

Wracając do wyrażenia $*$, mamy
\begin{align*}
\frac{d}{dt} \langle x \rangle = * &= \frac{i\hbar}{2m} \int \psi^* \left( x \nabla^2 \psi - \nabla^2 (x \psi) \right) d\vec{r} \\
&= -\frac{i\hbar}{m} \int \psi^* \frac{\partial \psi}{\partial x} \, d\vec{r}.
\end{align*}

Z definicji operatora pędu w kierunku $x$:

$$
\hat{p}_x = -i\hbar \frac{\partial}{\partial x},
$$

wynika, że

$$
\frac{d}{dt} \langle x \rangle = \frac{1}{m} \int \psi^* \left(-i\hbar \frac{\partial}{\partial x}\right) \psi \, d\vec{r} = \frac{1}{m} \langle \hat{p}_x \rangle.
$$

Analogicznie, pochodna wartości średniej pędu wyraża się przez

$$
\frac{d}{dt} \langle \hat{p}_x \rangle = - \left\langle \frac{\partial V}{\partial x} \right\rangle.
$$

Powyższe dwa równania stanowią treść twierdzenia Ehrenfesta i pokazują, że średnie wartości położenia i pędu w mechanice kwantowej zmieniają się zgodnie z klasycznymi równaniami ruchu. 
